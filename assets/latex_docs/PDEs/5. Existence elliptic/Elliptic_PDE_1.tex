\documentclass[12pt]{article}
\special{papersize=3in,5in}
\usepackage[utf8]{inputenc}
%PACKAGES
\usepackage{CJKutf8}
\usepackage[T1]{fontenc}
\makeatletter
\def\ps@pprintTitle{%
	\let\@oddhead\@empty
	\let\@evenhead\@empty
	\let\@oddfoot\@empty
	\let\@evenfoot\@oddfoot
}
\usepackage{amssymb,amsmath,physics,amsthm,xcolor,graphicx}
\usepackage[shortlabels]{enumitem}
\newtheorem{observation}{Observation}
\newtheorem{theorem}{Theorem}
\newtheorem{proposition}{Proposition}
\newtheorem{example}{Example}
\newtheorem{lemma}{Lemma}
\newtheorem{definition}{Definition}
\newtheorem{corollary}{Corollary}
\newtheorem{assumption}{Assumption}
\newcommand{\red}[1]{{\color{red}#1}}
\usepackage[colorlinks=true,
	linkcolor=blue,
	filecolor=magenta,
	urlcolor=cyan,
	pdfpagemode=FullScreen,]{hyperref}

\bibliographystyle{elsarticle-num}
\newcommand{\fk}[1]{\mathfrak{#1}}\newcommand{\wh}[1]{\widehat{#1}}\renewcommand{\Im}{\mathbf{Im}}
\newcommand{\br}[1]{\left\langle#1\right\rangle} \newcommand{\set}[1]{\left\{#1\right\}} \newcommand{\qp}[1]{\left(#1\right)}\newcommand{\qb}[1]{\left[#1\right]}
\newcommand{\Id}{\mathbf{I}}\renewcommand{\ker}{\mathbf{ker}}\newcommand{\supp}[1]{\mathbf{supp}(#1)}\renewcommand{\tr}[1]{\mathrm{tr}\left(#1\right)}
\renewcommand{\norm}[1]{\left\lVert #1 \right\rVert}\renewcommand{\abs}[1]{\left| #1 \right|}
\newcommand{\U}{_}\renewcommand{\star}{*}

\newcommand{\A}{\mathbb{A}}\newcommand{\C}{\mathbb{C}}\newcommand{\E}{\mathbb{E}}\newcommand{\F}{\mathbb{F}}\newcommand{\II}{\mathbb{I}}\newcommand{\K}{\mathbb{K}}\newcommand{\LL}{\mathbb{L}}\newcommand{\M}{\mathbb{M}}\newcommand{\N}{\mathbb{N}}\newcommand{\PP}{\mathbb{P}}\newcommand{\Q}{\mathbb{Q}}\newcommand{\R}{\mathbb{R}}\newcommand{\T}{\mathbb{T}}\newcommand{\W}{\mathbb{W}}\newcommand{\Z}{\mathbb{Z}}
\newcommand{\Aa}{\mathcal{A}}\newcommand{\Bb}{\mathcal{B}}\newcommand{\Cc}{\mathcal{C}}\newcommand{\Dd}{\mathcal{D}}\newcommand{\Ee}{\mathcal{E}}\newcommand{\Ff}{\mathcal{F}}\newcommand{\Gg}{\mathcal{G}}\newcommand{\Hh}{\mathcal{H}}\newcommand{\Kk}{\mathcal{K}}\newcommand{\Ll}{\mathcal{L}}\newcommand{\Mm}{\mathcal{M}}\newcommand{\Nn}{\mathcal{N}}\newcommand{\Pp}{\mathcal{P}}\newcommand{\Qq}{\mathcal{Q}}\newcommand{\Rr}{\mathcal{R}}\newcommand{\Ss}{\mathcal{S}}\newcommand{\Tt}{\mathcal{T}}\newcommand{\Uu}{\mathcal{U}}\newcommand{\Ww}{\mathcal{W}}\newcommand{\XX}{\mathcal {X}}\newcommand{\Zz}{\mathcal{Z}}
\renewcommand{\d}{\,\mathrm{d}}
\newcommand\restr[2]{\left.#1\right|_{#2}}
\newcommand{\qt}[1]{\left(#1\right)}

\pagestyle{empty}
\setlength{\parindent}{0in}

\begin{document}
\title{Elliptic PDE I}
\author{Liam Llamazares}
\date{5/21/2023}
\maketitle
\section{ Three line summary}
\begin{itemize}
  \item Elliptic partial differential equations (PDE) are PDE with no time variable and whose leading order derivatives satisfy a positivity condition.
  \item Using Lax Milgram's theorem we can prove existence and uniqueness of weak (distributional) solutions if the transport term is zero.
  \item If the transport term is non-zero solutions still exist but they are no longer unique and are determined by the kernel of the homogeneous problem.\end{itemize}
\section{Why should I care?}
Many problems arising in physics such as the Laplace and Poisson equation are elliptic PDE. Furthermore, the tools used to analyze them can be extrapolated to other settings such as
elliptic PDE. The analysis also helps contextualize and provide motivation for theoretical tools such as Hilbert spaces, compact operators and Fredholm operators.
\section{Notation}
Given $\lambda \in \C$ we will write $\overline{\lambda }$ for the conjugate of $\lambda $. Given a subset $A$ of some topological space it is also common to write  $\overline{A}$ for the closure of $A$. Though this is a slight abuse of notation we will do the same as the meaning will always be clear from context.

Given two topological vector spaces $X,Y$ we write  $\Ll(X,Y)$ for the space of continuous linear operators from $X$ to  $Y$.

We will use the Einstein convention that indices when they are repeated are summed over. For example we will write
\begin{align*}
  \nabla \cdot (a \nabla)=\sum_{i=1}^n \partial_i a_{ij} \partial _j =\partial_i a_{ij} \partial _j.
\end{align*}
Furthermore, we will fix $U \subset \R^n$ to be an open \textbf{bounded} (we will see later why this is important) set in $\R^n$.

We will use Vinogradov notation.

\section{Introduction}
Welcome back to the second post on our series of PDE. In the \href{https://nowheredifferentiable.com/2023-01-29-PDE-1/}{first post} of the series we dealt with the Fourier transform and it's application to defining spaces of weak derivatives and weak solutions to PDE. In this post we will consider an equation of the form
\begin{align}\label{PDE}
  \Ll u =f; \quad \restr{u}{\partial u} =0.
\end{align}
Where $\Ll$ is some differential operator, $f: U \to  \R$  is some known function and $u$ is the solution we want to find. As we shall soon see, the Sobolev spaces $H^s$ and the concept of weak solution will prove a natural setting for our analysis of elliptic PDE. We start right off with the definition
\begin{definition}
  Given $A: U \to \R^{d \times d}, b: U \to \R^d$ and $c:U \to \R$ we say that the differential operator
  \begin{align}\label{operator}
    \Ll:= -\nabla \cdot A \nabla + \nabla \cdot b +c\end{align}
  is \emph{elliptic} if there exists $\lambda >0$ such that
  \begin{align}\label{elliptic}
    \xi ^TA(x) \xi  \geq \lambda \abs{\xi }^2 , \quad\forall \xi \in \R^d , \quad\forall x \in U .
  \end{align}
\end{definition}
There are some points to clear up. Firstly, if this is the first time you've encountered  the ellipticity condition in \eqref{elliptic} then it may seem a bit strange.  Physically speaking, in a typical derivation of our PDE in \eqref{PDE}, $u$ is the density of some substance and $A$ corresponds to a diffusion matrix. Due to the ellipticity condition \eqref{elliptic} says that flow occurs from the region of \href{https://nowheredifferentiable.com/2023-12-23-PDEs-4-Physical_derivation_of_parabolic_and_elliptic_PDE/#:~:text=a)-,Diffusion,-%3A%20This%20is%20the}{higher to lower density}. Mathematically speaking \eqref{elliptic} will provide the necessary bound we need to apply Lax Milgram's theorem (ref section?).

This said, we have not yet defined which function space our coefficients live in and what $\Ll$ acts on. This is always a tricky aspect. ``How much can one get away with?''.So as to not make our lives too complicated in what remains we will make the following assumption
\begin{assumption}\label{Ass1}
  We assume that  $A_{ij}, b_i, c \in L^\infty (U)$ for all $i,j=1,\ldots,d$. Furthermore, $A$ is symmetric, that is  $A_{ij}=A_{ji}$.
\end{assumption}
T0 simplify the notation we will write for the bound on $a,b,c$
\begin{align*}
  \norm{a}_{L^\infty(U)}+\norm{b}_{L^\infty(U)}+\norm{c}_{L^\infty(U)}=M
\end{align*}

The first assumption will make it easy to get bounds on $\Ll$ and the second will be necessary to apply Lax Milgram's theorem and the third will prove useful when we look at the spectral theory of $\Ll$ (blogpost on this coming). With these assumptions we have the following
\begin{proposition}\label{domain L}
  The operator $\Ll$ defined in \eqref{operator} defines for all $s \in \R$ a bounded linear operator
  \begin{align*}
    \Ll : H_0^{s+2}\to H^s (U).
  \end{align*}
  \red{Have to add some stuff on trace and Sobolev on bounded domain.}
\end{proposition}
\begin{proof}
  We apply the usual trick of working first with a smooth functions $u$ that vanishes on the boundary. Then have that
  \begin{align*}
    \norm{\Ll u}_{H^s_0(U)}=\norm{\br{\xi }^s \wh{\Ll f}}_{L^2(U)}\lesssim  M \norm{\br{\xi }^{s+2}\wh{u}(\xi )}_{L^2(U)} =M \norm{u}_{H^{s+2}(U)} .
  \end{align*}
  Since $C_0^\infty(U)$ is dense in $H^k_0(U)$ for any $k \in \R$ we can extend $\Ll$ continuous to $H_0^{s+2}(U)$ by defining
  \begin{align*}
    \Ll u = \lim_{n \to \infty}\Ll u_n , \quad\forall u \in  H_0^{s+2}(U).
  \end{align*}
  Where $u_n \in C_0^\infty(U)$ is any sequence converging to $u$ in  $H^{s+2}(U)$.
\end{proof}
We now note that, by an integration by parts, if $u,v \in  C_0^\infty(U)$ then \begin{align*}
  \int_{U} \Ll u v =\int_{U}a \nabla u \cdot \nabla v + \int_{U} b \nabla  u v + \int_{U} cuv=: B(u,v)   .
\end{align*}
It is clear that $B$ is bilinear in an algebraic sense. Furthermore from Cauchy Schwartz and  the fact that
\begin{align*}
  \norm{u}_{H_1(U)}\sim \norm{u}_{L^2(U)}+\norm{\nabla u}_{L^2(U\to \R^d)}.
\end{align*}
Shows that we have the bound
\begin{align}\label{cont B}
  B(u,v)\lesssim M \norm{u}_{H_0^1(U)}\norm{v}_{H_0^1(U)}.
\end{align}
This allows us as in the previous proposition to extend $B$ from $C_0^\infty(U)$ to a continuous bilinear operator on  $H^1_0(U)$. We still have not mentioned what space $f$ should be in. We just saw that it makes sense to consider $u \in  H_0^1(U)$. Since by Proposition \ref{domain L} we have $\Ll u \in H^{-1}(U)$ and since we are looking for solutions to
\begin{align*}
  \Ll u=f.
\end{align*}
We see that we should impose $f \in H^{-1}(U)$. This can all be summarized as follows. holds
\begin{proposition}
  Given $f \in  H^{-1}(U)$, solving equation \eqref{PDE} (under assumption \ref{Ass1}) is equivalent to finding $u \in H_0^1(U)$ such that
  \begin{align}\label{reform}
    B(u,v)= (f,v) , \quad\forall v \in  H^{1}_0(U).
  \end{align}
\end{proposition}
Where we recall the ``duality notation''
\begin{align*}
  (f,v):= f(v) .
\end{align*}

\red{Labeling assumptions probably won't work. Check it.}
As a result we have reformulated our problem to something that looks very similar to the setup of Lax Milgram's theorem. In fact, if we suppose $b=0$ and $c \geq 0$ we are done
\begin{theorem}
  Suppose $b=0$  and $c \geq 0$. Then given $f \in  H^{-1}(U)$ there exists a unique solution $u$ to \eqref{reform} (and thus to \eqref{PDE}). Furthermore, the solution operator $\Ll^{-1}$ is a continuous operator	 $\Ll^{-1}: H^{-1}(U) \to H^1_0(U)$ with $\norm{\Ll^{-1}} \lesssim_U \lambda ^{-1}$.
\end{theorem}
\begin{proof}
  The continuity of $B$ was proved in  \eqref{reform}. It remains to see that $B$ is coercive. This follows from the fact that for smooth $u$
  \begin{align}\label{b=0}
    B(u,u) & = \int_{U}a \nabla u \cdot \nabla u + \int_{U} cu^2 \geq \lambda \norm{\nabla u}_{L^2(U \to \R^d)} \gtrsim_U \norm{u}_{H^1_0(U)}.
  \end{align}
  Where in first inequality we used the ellipticity assumption on $A$ and in the last inequality we used Poincaré's inequality.
\end{proof}
\red{Maybe best add $c \geq 0$ inn assumption here instead of $1$ if we later add on  $\alpha u$.}
Furthermore, by Rellich theorem $\Ll$ is compact and is self adjoint since $b$ is  $0$ so there is a countable basis of eigenvalues in  $L^2(U)$. Furthermore they must me smooth by Prop  $2$ and Sobolev embedding.

In the previous result, we somewhat unsatisfyingly had to assume that $ b$ was identically zero and had to impose the extra assumption  $c \geq 0$. These extra assumptions can be done away with, but at the cost of modifying our initial problem by a correction term $\gamma $ so we can once more obtain a coercive operator $B_\gamma $
\begin{theorem}[Modified problem]\label{mod}
  Given $f \in  H^{-1}(U)$ there exists some constant $\nu \geq 0$ (depending on the coefficients) such that for all $\gamma \geq \nu$  there exists a unique solution $u$ to
  \begin{align*}
    \Ll_\gamma u:= \Ll u+ \gamma u=f.
  \end{align*}
\end{theorem}
Furthermore $\Ll^{-1}: H^{-1}(U) \to H^1_0(U)$ is a continuous operator
\begin{proof}
  Once more, the proof will go through the Lax-Milgram theorem, where now we work with the bilinear operator $B_\gamma  $ associated to $\Ll_\gamma  $
  \begin{align*}
    B_\gamma  (u,v):= B(u,v) + \gamma  (u,v).
  \end{align*}
  The calculation proceeds in a similar fashion to  \eqref{b=0}, where now an additional application of Cauchy Schwartz to $\nabla u v = (\epsilon^{\frac{1}{2}} \nabla u)(\epsilon^{-\frac{1}{2}}v)$  shows that
  \begin{align*}
    B(u,u) & = \int_{U}a \nabla u \cdot \nabla u + \int_{U} b\cdot  \nabla u v +  \int_{U} cu^2 \geq \lambda \norm{\nabla u}_{L^2(U \to \R^d)}           \\
           & - \norm{b}_{L^\infty(U)} \qt{\epsilon \norm{\nabla u}_{L^2(U)}+ \epsilon ^{-1}\norm{u}_{L^2(U)}}- \norm{c}_{L^\infty(U)}\norm{u}_{L^2(U)} .
  \end{align*}
  Taking $\epsilon $ small enough (smaller than $\frac{1}{2} \lambda \norm{b}_{L^\infty(U)}^{-1}$ to be precise) and gathering up terms gives
  \begin{align}\label{b not 0}
    B(u,u) \geq \frac{\lambda}{2} \norm{\nabla u}_{L^2(U \to \R^d)} -\nu \norm{u}_{L^2(U)}.
  \end{align}
  Where we defined $\nu = \norm{b}_{L^\infty(U)} \epsilon ^{-1}+\norm{c}_{L^\infty(U)}$.The theorem now follows from the just proved \eqref{b not 0} and Poincaré's inequality as for all $\gamma \geq \nu$
  \begin{align*}
    B_\gamma (u,u)=B(u,u)+ \gamma \norm{u}_{L^2(U)} \geq\frac{\lambda}{2} \norm{\nabla u}_{L^2(U \to \R^d)}\gtrsim _U \norm{u}_{H_0^1(U)} .
  \end{align*}
\end{proof}
Now that we proved solutions for our modified problem $\Ll_\gamma $ it would be nice if we could somehow ``unmodify''. We reason as follows: we have that $u$ solves our original problem  \eqref{PDE} if and only if
\begin{align*}
  \Ll_\gamma u - \gamma  u =f.
\end{align*}
That is, moving $\gamma  u$ over and taking the inverse of $\Ll_\gamma $, if and only if
\begin{align*}
  \gamma^{-1}u -\Ll_\gamma ^{-1} u = \Ll_\gamma ^{-1}f.
\end{align*}
By what we just proved in Proposition \ref{mod} the operator $\Ll_\gamma^{-1}$, and thus $ \gamma \Ll_\gamma ^{-1}$ is a continuous operator form $H^{-1}(U)$ to  $H^{1}(U)$, as a result it is compact and we may apply the Fredholm alternative.

\red{Might have to restrict $f$ to  $L^2$ for some adjointness....}

\appendix


\bibliography{biblio.bib}
\end{document}
