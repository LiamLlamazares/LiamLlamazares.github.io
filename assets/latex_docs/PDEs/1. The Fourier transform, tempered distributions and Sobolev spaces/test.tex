% Options for packages loaded elsewhere
\PassOptionsToPackage{unicode}{hyperref}
\PassOptionsToPackage{hyphens}{url}
%
\documentclass[
]{article}
\usepackage{amsmath,amssymb}
\usepackage{lmodern}
\usepackage{iftex}
\ifPDFTeX
  \usepackage[T1]{fontenc}
  \usepackage[utf8]{inputenc}
  \usepackage{textcomp} % provide euro and other symbols
\else % if luatex or xetex
  \usepackage{unicode-math}
  \defaultfontfeatures{Scale=MatchLowercase}
  \defaultfontfeatures[\rmfamily]{Ligatures=TeX,Scale=1}
\fi
% Use upquote if available, for straight quotes in verbatim environments
\IfFileExists{upquote.sty}{\usepackage{upquote}}{}
\IfFileExists{microtype.sty}{% use microtype if available
  \usepackage[]{microtype}
  \UseMicrotypeSet[protrusion]{basicmath} % disable protrusion for tt fonts
}{}
\makeatletter
\@ifundefined{KOMAClassName}{% if non-KOMA class
  \IfFileExists{parskip.sty}{%
    \usepackage{parskip}
  }{% else
    \setlength{\parindent}{0pt}
    \setlength{\parskip}{6pt plus 2pt minus 1pt}}
}{% if KOMA class
  \KOMAoptions{parskip=half}}
\makeatother
\usepackage{xcolor}
\setlength{\emergencystretch}{3em} % prevent overfull lines
\providecommand{\tightlist}{%
  \setlength{\itemsep}{0pt}\setlength{\parskip}{0pt}}
\setcounter{secnumdepth}{-\maxdimen} % remove section numbering
\ifLuaTeX
  \usepackage{selnolig}  % disable illegal ligatures
\fi
\IfFileExists{bookmark.sty}{\usepackage{bookmark}}{\usepackage{hyperref}}
\IfFileExists{xurl.sty}{\usepackage{xurl}}{} % add URL line breaks if available
\urlstyle{same} % disable monospaced font for URLs
\hypersetup{
  pdftitle={The Fourier transform, tempered distributions and Sobolev spaces},
  pdfauthor={Liam Llamazares},
  hidelinks,
  pdfcreator={LaTeX via pandoc}}

\title{The Fourier transform, tempered distributions and Sobolev spaces}
\author{Liam Llamazares}
\date{10-14-2022}

\begin{document}
\maketitle

{UTF8}{gbsn}

\hypertarget{three-line-summary}{%
\section*{\texorpdfstring{ Three line summary
}{ Three line summary }}\label{three-line-summary}}
\addcontentsline{toc}{section}{ Three line summary }

\begin{itemize}
\item
  The \emph{Fourier transform} defines a linear isometry on the space of
  square-integrable complex-valued functions and has as inverse the
  \emph{inverse Fourier transform}(link). This allows us to write these
  functions as a superposition of harmonic functions.(link)
\item
  Linear differential operators act as a simple polynomial
  multiplication on the frequency domain.(link) Smooth functions have
  Fourier transforms that decay quickly and vice-versa(link). This
  correspondence also allows us to introduce fractional differential
  operators and pseudo differential operators such as
  \(\sqrt{-\Delta }\)(link).
\item
  The Sobolev spaces \(H^s\) and tempered distributions
  \({\mathcal S}'\) are complete spaces of functions to which we can
  extend differential operators.(link)
\end{itemize}

\hypertarget{introduction}{%
\section*{Introduction}\label{introduction}}
\addcontentsline{toc}{section}{Introduction}

Hi everyone, welcome to the first post in a new series focusing on
partial differential equations (PDEs). In this post, we'll introduce
some of the key analytical tools used to solve them. These tools will
play a prominent role in the future posts of this series and in other
series to come. We'll begin by introducing some notation and necessary
preliminaries on integration. Next, we will introduce the Fourier
transform, which is the fundamental object of harmonic analysis and
plays a fundamental role in partial differential equations (PDEs),
probability theory, and number theory. In particular, we will show in
future posts how many linear PDEs such as the heat equation can be
solved using it.\\
\strut \\
After this, we'll discuss the generalization of what it means to solve a
PDE. To do so we will need to introduce the space of (tempered)
distributions \({\mathcal S}'\) and Sobolev spaces \(H^s\). These two
spaces will be crucial in allowing us to extend the concept of solution
to PDE(link) to these spaces and use their completeness to prove the
existence, uniqueness, and regularity of solutions.\\
\strut \\
Much of the material here contained can be found in my master's
thesis(link) on the Navier Stokes equations which in turn are based on
Terence Tao's excellent notes(link).

\hypertarget{notation}{%
\subsection*{Notation}\label{notation}}
\addcontentsline{toc}{subsection}{Notation}

We write \(L^p({\mathbb R}^d)\) and
\(L^p({\mathbb R}^d\to\mathbb{C}^m)\) to denote the space of equivalence
classes of \(p\) integrable functions from \({\mathbb R}^d\) to
\({\mathbb R}\) and \(\mathbb{C}^m\) respectively. Given a function
defined on a factor space \(u:X\times Y\rightarrow Z\) we shall use
\(u(x)\) to stand for the slightly more cumbersome
\(u(x,\cdot ):Y\rightarrow Z\). Given
\(\alpha=(\alpha_1,...,\alpha_d)\in\mathbb{N}^d\) we shall write as is
standard
\[x^\alpha:=x^{\alpha_1}\cdots  x^{\alpha_d};\quad D^\alpha:=\partial_1^{\alpha_1}\cdots\partial_1^{\alpha_d}\]
and whenever the expression \(D^\alpha\) appears we will assume
implicitly that \(\alpha\in\mathbb{N}^d\). Given \(x\in{\mathbb R}^d\)
we will write \(\abs{x}\) to denote its norm and the \emph{Japanese
bracket} \(\left\langle x\right\rangle\) will signify
\[\left\langle x\right\rangle:=(1+\abs{x}^2)^{1/2}.\]

Finally, we will employ the notation \(f\lesssim g\) to mean that there
exists some constant \(C\) such that \(f\leq Cg\). If the value of \(C\)
depends on some other parameter such as the dimension \(d\) we shall
make this explicit by writing \(f\lesssim_d g\).

\hypertarget{some-useful-results-on-integration}{%
\subsection*{Some useful results on
integration}\label{some-useful-results-on-integration}}
\addcontentsline{toc}{subsection}{Some useful results on integration}

In PDEs, it is essential to be able to commute limits and differentials
with both summation and integration. The main way of doing this is the
dominated and monotone convergence theorem (link) together with the
following two propositions which we state in full generality for
functions valued in a Banach space \(E\). Though for now, we will only
need the case \(E={\mathbb R}\). On first reading, this can be assumed
for commodity.

\leavevmode\vadjust pre{\hypertarget{commutingux20limitsux20withux20integral}{}}%
\textbf{Proposition 1} (Continuity of the integral). \emph{Let
\((X,\mu )\) be a measure space, \(T\) a first countable metric space,
\((E,\norm{})\) a Banach space and \[f:T\times X\to E\] such that
\(f(t)\) is (Bochner(link)) integrable for all \({t}\in{T}\) and such
that \[\norm{f(t)}\leq g\in L^1(X)\quad \forall{t}\in {T}.\] Then given
\(t_0\in T\) we have that
\[\lim_{t\to t_0}\int_X f(t,x)\mu(dx) =\int_X\lim_{t\to t_0}f(t,x)\mu(dx) .\]
In consequence, if \(f(x)\) is continuous so is
\(\int_X f(t,x) \mu(dx)\).}

\emph{Proof.} The proof is an application of the dominated convergence
theorem(link) to the sequence of functions \(f_n:=f(t_n)\) where
\(\{t_n\}_{n=1}^\infty\) is some sequence converging to \(t\) and the
fact that, in first countable spaces, sequential continuity(link) is
equivalent to continuity.~◻

\leavevmode\vadjust pre{\hypertarget{derivationundertheintsign}{}}%
\textbf{Proposition 2} ({Differentiation under the integral sign}).
\emph{Let \(X\) be a measure space, \(U\) be a convex open interval of
\({\mathbb R}\), and \(E\) a Banach space. Consider \[f:U\times X\to E\]
such that:}

\begin{itemize}
\item
  \emph{\(f(t)\) is measurable for every \(t\in U\) and integrable for
  some \(t_0\in U\).}
\item
  \emph{For each \(x\in X\) we have that \(f(x)\) is differentiable and
  there exists an integrable function \(g: X\to{\mathbb R}\) such that
  \[\norm{\partial_{t}{f}(t,x)}\leq g(x)\quad \forall{(t,x)}\in {U\times E}.\]
  Then it holds that
  \[\partial_{t}{\int_X f(t)\mu(dx)}=\int_X \partial_{t}f(t) \mu(dx).\]}
\end{itemize}

\emph{Proof.} First, we show that the integral on the left-hand side of
the equation above makes sense. Let \(t\in U\) be any, then by the mean
value inequality we have that
\[\frac{f(t)-f(t_0)}{t-t_0}\leq \sup_{s \in U}\norm{\partial _t f(s)}\leq g \in L^1(X).\]
Thus, we may apply the previous proposition to commute the limit with
the integral \[\begin{gathered}
        \lim_{t \to t_0}\frac{\int_{X} f(t)\mu (dx)-\int_{X}f(t_0) \mu (dx)}{t-t_0}  =\lim_{t \to t_0}\int_{X}    \frac{f(t)-f(t_0)}{t-t_0} \mu (dx) \\=\int_{X}\lim_{t \to t_0} \frac{f(t)-f(t_0)}{t-t_0} \mu (dx) =\int_{X}\partial _t f(t) \mu(dx).
    \end{gathered}\] Where in the first equality we also used the
linearity of the Bochner integral (link) and in the last equality we
used that by hypothesis \(f(x)\) is everywhere differentiable. This
concludes the proof.~◻

In the literature the usual hypothesis is that \(\partial _t f\) is
integrable everywhere (see for example \cite{FF} page 108), however as
we have seen it is sufficient to require the integrability at a point.

\hypertarget{the-fourier-transform-on-mathbb-rd}{%
\section{\texorpdfstring{The Fourier transform on
\({\mathbb R}^d\)}{The Fourier transform on \{\textbackslash mathbb R\}\^{}d}}\label{the-fourier-transform-on-mathbb-rd}}

These preliminaries out of the way, we now get to our first main
section. We assume the reader has encountered the Fourier transform
before, and limit ourselves to giving a rundown of the theory and main
ideas, taking the reader through a basic definition to an extension of
it. To make the Fourier transform an isometry we will use the following
convention.

\textbf{Definition 1}. \emph{Given
\(f\in L^1({\mathbb R}^d\to\mathbb{C}^m)\) the \emph{Fourier transform}
of \(f\) and the \emph{inverse Fourier transform of} \(f\) are defined
respectively by
\[\hat{f}(\xi):=\int_{{\mathbb R}}^d}f(x)e^{-2\pi i\xi\cdot  x} dx, \quad \check{f}(\xi):=\int_{{\mathbb R}}^d}f(x)e^{2\pi i\xi\cdot  x} dx\]}

Other exponents such as \(e^{-ix}, e^{-\pi i x}\) can also be used with
identical results modulo constants. Let us introduce the notation
\[\mathcal{F}_1 f:=\hat{f}, \mathcal{F}_1^*f\] for the Fourier transform
(on \(L^1\)) and its inverse. The following proposition holds.

\leavevmode\vadjust pre{\hypertarget{regularitydecayft}{}}%
\textbf{Proposition 3} ( \textbf{Properties of \(\mathcal{F}_1\)}).
\emph{Given \(f\in L^1({\mathbb R}^d\to\mathbb{C}^m)\)}

\begin{enumerate}
\item
  \emph{The Fourier transform \(\mathcal{F}_1\) is a continuous linear
  operators
  \[\mathcal{F}_1:L^1({\mathbb R}^d\to\mathbb{C}^m) \to L^\infty({\mathbb R}^d\to\mathbb{C}^m)\].}
\item
  \emph{If \(x^\alpha f(x)\in L^1({\mathbb R}^d\to\mathbb{C}^m)\)
  then:\[D^\alpha \hat{f}(\xi)=(-2\pi i)^\abs{\alpha}\widehat{x^{\alpha} f}(\xi)\qquad\forall \xi\in{\mathbb R}^d.\]}
\item
  \emph{If \(f\) is absolutely continuous in \(x_j\) for almost every
  \(x_1,...x_{j-1},x_{j+1},...,x_d\) then then
  \[\widehat{\partial x_j f}(\xi)=2\pi i\xi_j\widehat{ f}(\xi)\qquad\forall \xi\in{\mathbb R}^d.\]}
\end{enumerate}

\emph{Proof.} The proof of the first point is an application of the
triangle inequality as
\[\abs{\hat{f}(\xi )}=\abs{\int_{{\mathbb R}}^d}f(x)e^{2\pi i x\cdot \xi } dx}\leq\int_{{\mathbb R}}^d}\abs{f(x)}dx\quad\forall \xi \in {\mathbb R}^d.\]
and similarly for the inverse. The second point follows by
differentiation under the integral(link). The third point can be proved
by using integration by parts and the observation that every absolutely
continuous function must go to zero at infinity.~◻

We note that the analogous holds for the inverse Fourier transform where
it is only necessary to remove the minus sign in \(2\) and a change of
the sign in \(3\). The proof can be completed in an identical fashion or
by noting that \(\check{f}(\xi )=\hat{f}(-\xi )\). The importance of
these last two properties is that they give us information on the
Fourier transform of a function \(f\) based on its regularity and decay.
They can be stated informally as decay gives regularity and regularity
gives decay (as the Fourier transform of any
\(f\in L^1({\mathbb R}^d\to{\mathbb R}^d)\) decays to 0 at infinity by
the Riemann-Lebesgue lemma(link)). In general, the Fourier transform of
an integrable function is not itself integrable. That is,
\(L^1({\mathbb R}^d\to\mathbb{C}^m)\) is not closed under the Fourier
transform. We now introduce a space that is closed under the Fourier
transform; the Schwartz space, which can be thought of as the space of
infinitely regular functions with infinite decay:

\leavevmode\vadjust pre{\hypertarget{Schwartzux20spaceux20definition}{}}%
\textbf{Definition 2}. \emph{The \emph{Schwartz space}
\(\mathcal{S}({\mathbb R}^d\to\mathbb{C}^m)\) is:
\[\mathcal{S}({\mathbb R}^d\to\mathbb{C}^m):=\lbrace f\in\mathbb{C}^\infty({\mathbb R}^d\to\mathbb{C}^m):x^\alpha D^\beta f\in L^\infty({\mathbb R}^d\to\mathbb{C}^m)\quad\forall\hspace{2pt}\alpha,\beta\in \mathbb{N}^d\rbrace.\]}

By applying Proposition \(1\) (link) we deduce that, given
\(f\in\mathcal{S}({\mathbb R}^d\to\mathbb{C}^m)\) (linkme1)
\[\begin{aligned}
\label{rdft}
    \xi^\alpha D^\beta \hat{f}(\xi)   & =(-2\pi i)^{\abs{\beta}}\xi^{\alpha}\widehat{x^{\beta}f}=\frac{(-2\pi i)^{\abs{\beta}}}{(2\pi i)^{\abs{\alpha}}}\widehat{D^{\alpha}x^{\beta}f}\in L^\infty({\mathbb R}^d\to\mathbb{C}^m) \\
    \xi^\alpha D^\beta \check{f}(\xi) & =\frac{(2\pi i)^{\abs{\beta}}}{(-2\pi i)^{\abs{\alpha}}}\widehat{D^{\alpha}x^{\beta}f}\in L^\infty({\mathbb R}^d\to\mathbb{C}^m).\end{aligned}\]
Hence, the Fourier transform and the inverse Fourier transform restrict
to endomorphisms of the Schwartz space which we will denote respectively
by
\[\mathcal{F}_\mathcal{S},\mathcal{F}^*_\mathcal{S}:\mathcal{S}({\mathbb R}^d\to\mathbb{C}^m)\to\mathcal{S}({\mathbb R}^d\to\mathbb{C}^m).\]
The next item on the agenda is Plancherel's theorem which is proven via
the following lemma, in which we shall use the
notation\[\left\langle f,g\right\rangle_{L^2({\mathbb R}^d\to\mathbb{C}^m)}:=\int_{{\mathbb R}}^d}f(x)\cdot  \overline{g(x)} dx\]
for the inner product on \(L^2({\mathbb R}^d\to\mathbb{C}^m)\).

\leavevmode\vadjust pre{\hypertarget{Planncherellemma}{}}%
\textbf{Lemma 1} (\textbf{Plancherel for Schwartz functions}).
\emph{Given \(f,g\in \mathcal{S}({\mathbb R}^d\to\mathbb{C}^m)\) (i)
\(\left\langle\mathcal{F}_\mathcal{S}f,g\right\rangle_{L^2({\mathbb R}^d\to\mathbb{C}^m)}=\left\langle f,\mathcal{F}_\mathcal{S}^*g\right\rangle_{L^2({\mathbb R}^d\to\mathbb{C}^m)};\quad\left\langle\mathcal{F}_\mathcal{S}^*f,g\right\rangle_{L^2({\mathbb R}^d\to\mathbb{C}^m)}=\left\langle f,\mathcal{F}_\mathcal{S}g\right\rangle_{L^2({\mathbb R}^d\to\mathbb{C}^m)}\)
(ii)
\(\mathcal{F}_\mathcal{S}^*\mathcal{F}_\mathcal{S}f=\mathcal{F}_\mathcal{S}\mathcal{F}_\mathcal{S}^*f=f\)}

The first point can be proved by a direct application of Fubini. The
second point is trickier and is the crux of why we call
\(\mathcal{F}^*\) the \emph{inverse} Fourier transform. The proof can be
found in \cite{Tay} pages 222-226. From \((i)\) and \((ii)\) we deduce
immediately that, given a Schwartz function \(f\), \[\label{Plancherel}
    \norm{f}_{L^2({\mathbb R}^d\to\mathbb{C}^m)}=\norm{\mathcal{F}_\mathcal{S}f}_{L^2({\mathbb R}^d\to\mathbb{C}^m)}=\norm{\mathcal{F}_\mathcal{S}^*f}_{L^2({\mathbb R}^d\to\mathbb{C}^m)}\]
That is, the restrictions of \(\mathcal{F}_1\) and \(\mathcal{F}^*_1\)
\[\mathcal{F}_\mathcal{S},\mathcal{F}_\mathcal{S}^*:\mathcal{S}({\mathbb R}^d\to\mathbb{C}^m)\to\mathcal{S}({\mathbb R}^d\to\mathbb{C}^m)\]
are linear unitary operators which are the inverse the one to the other.
We obtain the following result.

\leavevmode\vadjust pre{\hypertarget{Plancherelux27sux20Theorem}{}}%
\textbf{Proposition 4} (\textbf{Plancherel's theorem}).
\emph{\(\mathcal{F}_\mathcal{S}\) and \(\mathcal{F}_\mathcal{S}^*\) may
be extended to unitary operators:
\[\mathcal{F},\mathcal{F}^*:L^2({\mathbb R}^d\to\mathbb{C}^m)\to L^2({\mathbb R}^d\to\mathbb{C}^m)\]
With \(\mathcal{F}\mathcal{F}^*=\mathcal{F}^*\mathcal{F}=Id\).}

\emph{Proof.} This is a immediate result of the density of
\(\mathcal{S}({\mathbb R}^d\to\mathbb{C}^m)\) in the larger
\(L^2({\mathbb R}^d\to\mathbb{C}^m)\) and the completeness of
\(L^2({\mathbb R}^d\to\mathbb{C}^m)\) together with \((ii)\) and
Plancherel (link). As, by a simple limiting argument, it is easy to see
that continuous extensions of continuous linear operators preserve the
norm and the inverses of the operators in question.~◻

It is reasonable to wonder if this extended Fourier transform
\(\mathcal{F}\) coincides with our initial definition when reasonable.
That is, whether given a function
\(f \in L^1({\mathbb R}^d\to\mathbb{C}^m)\cap L^2({\mathbb R}^d\to\mathbb{C}^m)\)
we have that
\[\mathcal{F}f(\xi)=\mathcal{F}_1f(\xi)=\int_{{\mathbb R}}^d}f(x)e^{-2\pi ix\cdot \xi} dx\qquad\forall \xi\in{\mathbb R}^d.\]
This indeed holds as can be seen by taking a sequence of functions
\(\lbrace f_n\rbrace_{n=1}^\infty\in{\mathcal S}({\mathbb R}^d\to\mathbb{C}^m)\)
converging to \(f\) in \(L^1({\mathbb R}^d\to\mathbb{C}^m)\) and in
\(L^2({\mathbb R}^d\to\mathbb{C}^m)\). As then \(\mathcal{F}_1 f_n\)
converges uniformly to \(\mathcal{F}_1f\) by \(1\) (point one prop 1
link) and in \(L^2({\mathbb R}^d\to\mathbb{C}^m)\) to \(\mathcal{F}f\)
by Proposition 2 (link), from where we deduce that as desired
\(\mathcal{F}_1f=\mathcal{F}f\).

\hypertarget{the-fourier-transform-of-periodic-functions}{%
\section{The Fourier transform of periodic
functions}\label{the-fourier-transform-of-periodic-functions}}

We now extend our previous results to \({\mathbb Z}^d\) periodic
functions, though, for any period, analogous results may be achieved. By
identifying \({\mathbb Z}^d\) periodic functions with functions on the
torus \({\mathbb T}^d:={\mathbb R}^d/{\mathbb Z}^d\equiv [0,1]^d\) we
will work with functions \(f:{\mathbb T}^d\to\mathbb{C}^m.\) As we will
see strikingly similar results are achieved in this setting

\textbf{Definition 3}. \emph{Given
\(f\in L^1({\mathbb T}^d\to\mathbb{C}^m)\) we define the \emph{\(k\)-th
Fourier coefficient} of \(f\) as:
\[\hat{f}(k):=\int_{{\mathbb T}}^d}f(x)e^{-2\pi ik\cdot x}dx.\]}

We thus obtain a function \(\hat{f}\) on \({\mathbb Z}^d\) which we
shall call the Fourier series of \(f\) and a continuous linear function
which we shall denote as in the euclidean case:
\[\mathcal{F}_1:L^1({\mathbb T}^d\to\mathbb{C}^m)\to l^\infty({\mathbb Z}^d\to\mathbb{C}^m)\]
where \(l^{\infty}({\mathbb Z}^d\to\mathbb{C}^m)\) is the set of bounded
sequences from \({\mathbb Z}^d\) to \(\mathbb{C}^m\). As before we have
the following result:

\leavevmode\vadjust pre{\hypertarget{regdecaypft}{}}%
\textbf{Proposition 5}. \emph{Given
\(f\in L^1({\mathbb T}^d\to\mathbb{C}^m)\) if \(f\) is absolutely
continuous in \(x_j\) for almost every
\(x_1,...x_{j-1},x_{j+1},...,x_d\) then
\[\widehat{\partial_{x_j}{f}}(k)=2\pi i k_j\widehat{f}(k)\qquad\forall k\in{\mathbb Z}^d.\]}

In particular if \(f\in C^\infty({\mathbb T}^d\to\mathbb{C}^m)\) we have
that: \[\label{rgivesdpft}
    \widehat{D^\alpha f}(k)=(2\pi ik)^\alpha\hat{f}(k),\] and as a
result we have that \(\hat{f}\) is of rapid decrease (i.e. \(\hat{f}\)
decreases faster than the inverse of any polynomial). We thus have, as
before, an induced map:
\[\mathcal{F}_{C^\infty}:C^\infty({\mathbb T}^d\to\mathbb{C}^m)\to s({\mathbb Z}^d\to\mathbb{C}^m)\]
\[f\mapsto \hat{f}.\] Where \(s({\mathbb Z}^d\to\mathbb{C}^m)\) are the
sequences from \({\mathbb Z}^d\) to \(\mathbb{C}^m\) that are of rapid
decrease (this space now plays the role of the Schwartz space
\({\mathcal S}({\mathbb R}^d\to\mathbb{C}^m)\)). Similarly to the
non-periodic case we now define
\[\mathcal{F}^*_{C^\infty}:s({\mathbb Z}^d\to\mathbb{C}^m)\to C^\infty({\mathbb T}^d\to\mathbb{C}^m);\quad a\mapsto\check{a}\]

with: \[\check{a}(x):=\sum_{k\in{\mathbb Z}^d}a(k)e^{2\pi ik\cdot x}.\]
This is indeed smooth as, by the rapid decay of \(a\), we may
differentiate under the integral sign(link) (we recall that sums are
just integrals against discrete measures) to commute all derivatives
with the above sum to obtain that \[\label{dgivesrpft}
    D^\alpha \check{a}(x)=\sum_{k\in{\mathbb Z}^d}(2\pi ik)^\alpha a(k)e^{2\pi ik\cdot x}\quad\forall{a}\in {s({\mathbb Z}^d\to\mathbb{C}^m)}.\]
It is now possible to prove as with the euclidean case that
\[\mathcal{F}_{C^\infty}\mathcal{F}^*_{C^\infty}=\mathcal{F}^*_{C^\infty}\mathcal{F}_{C^\infty}=Id,\]
and that analogously given \(f\) smooth, and \(a\) of rapid decrease
\[\left\langle\mathcal{F}_{C^\infty} f,a\right\rangle_{l^2({\mathbb Z}^d\to\mathbb{C}^m)}=\left\langle f,\mathcal{F}_{C^\infty}^*a\right\rangle_{L^2({\mathbb R}^d\to\mathbb{C}^m)};\quad\left\langle\mathcal{F}_{C^\infty}^* a,f\right\rangle_{L^2({\mathbb R}^d\to\mathbb{C}^m)}=\left\langle a,\mathcal{F}f\right\rangle_{l^2({\mathbb Z}^d\to\mathbb{C}^m)}.\]
See for example \cite{Tay} pages 197-206. We conclude that
\(\mathcal{F}_{C^\infty}\) are unitary linear functions and that hence:

\leavevmode\vadjust pre{\hypertarget{Plancherelperiodictheorem}{}}%
\textbf{Proposition 6} (\textbf{Plancherel's (periodic) theorem}).
\emph{\(\mathcal{F}_{C^\infty}\) and \(\mathcal{F}_{C^\infty}^*\) may be
extended to unitary operators:
\[\mathcal{F}:L^2({\mathbb T}^d\to\mathbb{C}^m)\to l^2({\mathbb T}^d\to\mathbb{C}^m);\quad \mathcal{F}^*:l^2({\mathbb Z}^d\to\mathbb{C}^m)\to L^2({\mathbb R}^d\to\mathbb{C}^m)\]
with \(\mathcal{F}\mathcal{F}^*=\mathcal{F}^*\mathcal{F}=Id.\)}

We note that, as for the Euclidean case, given
\(f\in L^2({\mathbb T}^d\to\mathbb{C}^m)\cap L^1({\mathbb T}^d\to\mathbb{C}^m)\)
by an identical argument it holds that
\[\mathcal{F}f(k)=\mathcal{F}_1f(k)=\int_{{\mathbb R}}^d}f(x)e^{-2\pi ik\cdot x}\]
and where now Plancherel's theorem gives that, for such \(f\):
\[\label{Plancerelpft}
    f(x)=\sum_{k\in{\mathbb Z}^d}\hat{f}(k)e^{2\pi ik\cdot x}.\]

\hypertarget{distributions-and-sobolev-spaces}{%
\section{Distributions and Sobolev
Spaces}\label{distributions-and-sobolev-spaces}}

Here we will quickly recall the concepts of tempered distributions and
Sobolev spaces, which are concepts of utmost importance in the field of
PDEs and Fourier analysis. The idea is as follows, given a topological
vector space \(V\) we denote the dual of \(V\) by
\[V':=\lbrace w:E\to\mathbb{C}\hspace{2pt}: w \hspace{2pt}\text{continuous}\rbrace.\]
Furthermore, given \(w\in V'\) and \(u\in V\) we use the notation
\[(u,w):=w(u)\] The reason for this is that \(w(u)\) can often be
interpreted as the inner product of \(w\) against \(u\) in a suitable
Hilbert space. In our case, this Hilbert space will be
\(L^2({\mathbb R}^d\to\mathbb{C})\) and the above interpretation will
allow us to extend differentiation and the Fourier transform by a
technique called ``duality''. We will apply this where
\(V={\mathcal S}({\mathbb R}^d\to\mathbb{C})\), first we need to
introduce a topology on \(E\). In general, given a real vector space
\(V\) together with a countable family of semi-norms
\(\lbrace p_j\rbrace_{j=0}^\infty\) with the property that: given
\(x\neq 0\), there exists \(j\) such that \(p_j(x)\neq 0\). Then
\[\label{seminormsgivemetric}
    d(x,y):=\sum_{j=0}^\infty 2^{-j}\frac{p_j(x-y)}{1+p_j(x-y)}\quad \forall{x,y}\in {E}\]
is a translation invariant metric on \(E\). In the case of the Schwartz
space \({\mathcal S}({\mathbb R}^d\to\mathbb{C})\) we give it the
topology induced by \[\label{seminornormsSchwartz}
    p_k(u):=\sum_{\abs{\alpha}\leq k} \sup_{x\in{\mathbb R}^d}\left\langle x\right\rangle^k \abs{D^\alpha u(x)}.\]
Though other families of semi-norms the reader may be familiar with such
as
\[p_{k,\alpha}:=\sup_{x\in{\mathbb R}^d}\abs{x}^k \abs{D^\alpha u(x)};\quad\text{or}\quad p'_{k,\alpha}:=\sup_{x\in{\mathbb R}^d}(1+\abs{x})^k \abs{D^\alpha u(x)}\]
induce equivalent topologies. We note that with this metric
\({\mathcal S}({\mathbb R}^d\to\mathbb{C})\) is complete. For a quick
proof based on the fundamental theorem of calculus see for example
\cite{Foll} page 237.

\textbf{Definition 4}. \emph{The space of \emph{tempered distributions}
is the dual space to \({\mathcal S}({\mathbb R}^d\to\mathbb{C})\) with
the topology generated by \(p_k)\). We write it
\({\mathcal S}'({\mathbb R}^d\to\mathbb{C})\).}

One may verify that we have the inclusion \[\begin{aligned}
\label{lpisdistr}
    L^p({\mathbb R}^d\to\mathbb{C})\hookrightarrow{\mathcal S}'({\mathbb R}^d\to\mathbb{C});\quad f\mapsto T_f\end{aligned}\]
where given \(u\in {\mathcal S}({\mathbb R}^d\to\mathbb{C})\) we define
\[T_f(u):=\left\langle u,f\right\rangle:=\int_{{\mathbb R}}^d} u\overline{f}.\]
Let us write \(T^t\) for the transpose of a linear function \(T\) and
recall that the Fourier transform is an endomorphism of the Schwartz
space. Then, given two Schwartz functions \(u,v\) we have already seen
that, by a simple application of Fubini,
\[T_{\mathcal{F}v}(u)=\left\langle u,\mathcal{F}v\right\rangle=\left\langle\mathcal{F}^{-1} u,v\right\rangle=(\mathcal{F}^{-1}u,T_v)\]
and integration by parts gives
\[T_{D^\alpha v}(u)=\left\langle u,D^\alpha v\right\rangle=(-1)^\abs{\alpha}\left\langle D^\alpha u,v\right\rangle=((-1)^\abs{\alpha}D^\alpha u,T_v)\quad\forall{\alpha}\in {\mathbb{N}^d}.\]

This gives us a way of extending the Fourier transform and
differentiation to the space of tempered distributions.

\textbf{Definition 5}. \emph{Given
\(w\in\mathcal{S}'({\mathbb R}^d\to\mathbb{C})\) and
\(\alpha\in\mathbb{N}^d\) we define the \emph{(distributional) Fourier
transform} of \(w\) by \[\mathcal{F}w:= w\circ \mathcal{F}^{-1}\] and
the \emph{(weak) \(\alpha\)'th} derivative of \(w\) by
\[D^\alpha w:= w\circ((-1)^\abs{\alpha}D^\alpha).\]}

The method we used above to extend \(\mathcal{F}, D^\alpha\) is called
the \emph{duality method} and appears very frequently. Another way of
writing the above definition is:
\[(u, \mathcal{F}w):=(\mathcal{F}^{-1} u, w);\quad(u,D^\alpha w):=(-1)^\abs{\alpha}(D^\alpha u,w)\]
Due to our previous discussion, we have that with this definition
\[\label{ftlpdistr}
    \mathcal{F}T_u=T_{\mathcal{F}u};\quad D^\alpha T_{u}=T_{D^\alpha u}\quad\forall u\in {\mathcal S}({\mathbb R}^d).\]
Just as one would desire.

Two other operations that are permitted on
\({\mathcal S}({\mathbb R}^d\to\mathbb{C})\) are multiplication by
functions of polynomial growth \(p\) and the application of the inverse
Fourier transform which we shall, as for \(L^2\) functions, denote by
\(\mathcal{F}^{-1}\). Both definitions are once again being given by
duality.
\[(u,pw):=(\overline{p}u,w);\quad (u,\mathcal{F}^{-1}w):=(\mathcal{F}u,w) .\]

Before ending our discussion of (scalar) tempered distributions we
comment on some generalizations. We first note that the previous
discussion works equivalently for vector-valued distributions, i.e.
elements of the dual to \(\mathcal{S}({\mathbb R}^d\to\mathbb{C}^m)\),
which we shall denote by\[\mathcal{S}'({\mathbb R}^d\to\mathbb{C}^m)\]
where the only change is that the inclusion of integrable functions is
now given by \[\begin{aligned}
\label{lpisvectordistr}
    L^p({\mathbb R}^d\to\mathbb{C}^m)\hookrightarrow{\mathcal S}'({\mathbb R}^d\to\mathbb{C}^m);\quad f\mapsto T_f\end{aligned}\]
with \(T_f\) defined by (linkme2)
\[T_f(u):=\int_{{\mathbb R}}^d} u\cdot \bar{f}\quad\forall{u}\in {{\mathcal S}({\mathbb R}^d\to\mathbb{C}^m)}.\]
In both cases we have that, by duality, due to the formulas derived
previously (link1), given a tempered distribution \(w\) and
\(\alpha\in \mathbb{N}^d\) \[\begin{aligned}
\label{derivativedistronperiodic}
    D^\alpha\mathcal{F}w =(-2\pi i)^{\abs{\alpha}}\mathcal{F}x^\alpha w;\quad
    \mathcal{F}D^{\alpha} w= (2\pi i)^{\abs{\alpha}}x^\alpha\mathcal{F}w.\end{aligned}\]
As we have observed before (link) multiplication by functions of
polynomial growth is a well-defined operation on
\({\mathcal S}({\mathbb R}^d\to\mathbb{C}^m)\) so the above expressions
are also well defined. A quick verification also shows that, since
Plancherel's theorem holds for all Schwartz functions,
\[\label{planchereldistr}
    \mathcal{F}^{-1}\mathcal{F}w=\mathcal{F}\mathcal{F}^{-1}w=w.\]

Finally, in addition to ``changing the image'' of our distributions, we
may also ``change the domain'' by considering, for example, periodic
tempered distributions. Where now, as we saw in the section on the
Fourier transform, \(C^\infty({\mathbb T}^d\to\mathbb{C}^m)\) takes the
place of the Schwartz space and where we place on
\(C^\infty({\mathbb T}^d\to\mathbb{C}^m)\) the topology defined by the
countable family of semi-norms:
\[q_k(u):=\sum_{\abs{\alpha}\leq k}\sup_{x\in{\mathbb T}^d}\abs{D^\alpha u}\]
and denote its dual by \({\mathcal S}'({\mathbb T}^d\to\mathbb{C}^m).\)
Note that, as the domain is bounded, multiplication by polynomials to
ensure rapid decrease is redundant. By defining as is natural the
Fourier series of a periodic distribution \(w\) by the sequence (which
can be shown to be of polynomial growth)
\[\label{fouriercoeffperiodicdist}
    \hat{w}(k):=(e^{-2\pi ikx},w)\quad k\in{\mathbb Z}^d\] and its
\(\alpha\)-th distributional derivative by
\[(u,D^\alpha w):=(-1)^\abs{\alpha}(D^\alpha u,w)\] we derive formulas
analogous to the ones seen in the section on the Fourier transform for
``periodic" distributions as well. Namely: \[\label{derivativedistr}
    w=\sum_{k\in{\mathbb Z}^d}\hat{w}(k)e^{2\pi i k x};\quad\widehat{D^\alpha w}(k)=(2\pi ik)^\alpha\hat{w}(k).\]

To prove it all we have to do is apply Plancherel on \(u\) and move
terms around via duality, \[\begin{gathered}
    (u,w)=(\sum_{k\in {\mathbb Z}^d}\hat{u}(k)e^{2\pi ik\cdot x},w )=\sum_{k\in {\mathbb Z}^d}  \hat{u}(k)\hat{w}(k)\\=\sum_{k\in {\mathbb Z}^d}  \left(\int_{{\mathbb T}}^d} u(x)e^{-2\pi i\omega\cdot  x}dx\right)\hat{w}(k)=\int_{{\mathbb T}}^d} u(x)\left(\sum_{k\in {\mathbb Z}^d} \hat{w}(k) e^{-2\pi i\omega\cdot  x}\right)dx.\end{gathered}\]
This proves the first part and the second can be proved directly by
considering the relevant definitions.\\
\strut \\
To sum up what we have seen, due to the natural inclusion of integrable
functions in the space of tempered distributions and the analogous
inclusion in the periodic case, the notion of Fourier transform and
differentiation extends to the larger space of tempered distributions.
This allows us to manipulate rough functions (periodic or non-periodic)
as if they had Fourier transforms and were smooth. As we shall see, this
will prove of great use when obtaining ``distributional solutions'' to
some PDEs. We now give the general method by which this achieved

\textbf{Definition 6}. \emph{Consider a mapping
\[P:A\subset{\mathcal S}({\mathbb R}^d\to\mathbb{C}^m)\to {\mathcal S}({\mathbb R}^d\to\mathbb{C}^m)\]
that extends to
\[P:S\subset {\mathcal S}'({\mathbb R}^d\to\mathbb{C}^n)\to {\mathcal S}'({\mathbb R}^d\to\mathbb{C}^n)\]
where \(S\) is some subset of
\({{\mathcal S}'}({\mathbb R}^d\to\mathbb{C}^m)\) containing
\({\mathcal S}^\infty({\mathbb R}^d\to\mathbb{C}^m)\). Then, given
\(f \in {\mathcal S}'({\mathbb R}^d\to \mathbb{C}^n)\) we say that a
\emph{distributional solution} to \(Pw=f\) is any tempered distribution
\(w\in S\) verifying \(Pw=f\).}

In the above definition, \(P\) typically defines a linear or non-linear
differential equation. Note that the above definition may be extended
without any difficulty to the case of (periodic) distributional
solutions in the case where
\(P: C^\infty({\mathbb T}^d\to\mathbb{C}^m)\to C^\infty({\mathbb T}^d\to{\mathbb R}^n)\).

\textbf{Example 1}. \emph{Set \(P=\Delta\), as we have seen previously
(link) \(\Delta\) extends to
\(S:={\mathcal S}'({\mathbb R}^d\to\mathbb{C}^m)\) with
\[\mathcal{F}(\Delta \omega)=-4\pi \abs{\xi }^2\mathcal{F}w .\] As a
result we deduce that for any
\(f \in {\mathcal S}'({\mathbb R}^d\to \mathbb{C}^n)\) the equation
\(Pw=f\) has as it's unique solution
\[w=\mathcal{F}^{-1}\left(\frac{-f}{4\pi \abs{\xi }^2 }\right).\]}

\hypertarget{sobolev-spaces}{%
\section{Sobolev spaces}\label{sobolev-spaces}}

Sobolev spaces form a particular case of tempered distributions that we
interpret as being smooth and integrable up to sufficient orders
(linkme3).

\leavevmode\vadjust pre{\hypertarget{sobolevdef}{}}%
\textbf{Definition 7}. \emph{Given \(k\in\mathbb{N}^+\) we define the
\emph{Sobolev space} \(H^k({\mathbb R}^d\to\mathbb{C}^m)\) as:
\[\begin{gathered}
        H^k({\mathbb R}^d\to\mathbb{C}^m):=\\ \lbrace f\in L^2({\mathbb R}^d\to\mathbb{C}^m): D^\alpha f\in L^2({\mathbb R}^d\to\mathbb{C}^m)\hookrightarrow{\mathcal S}'({\mathbb R}^d\to\mathbb{C}^m)\quad\forall\hspace{2pt}\abs{\alpha}\leq k\rbrace.
    \end{gathered}\] Where, we consider
\(L^2({\mathbb R}^d\to\mathbb{C}^m)\) as a subspace of
\({\mathcal S}'({\mathbb R}^d\to\mathbb{C}^m)\) (see linkto2).}

We may interpret the Sobolev space \(H^k({\mathbb R}^d\to\mathbb{C}^m)\)
as the space of \(k\) times differentiable functions in
\(L^2({\mathbb R}^d\to\mathbb{C}^m)\) and we give it the norm:
\[\norm{f}_{H^k({\mathbb R}^d\to\mathbb{C}^m)}:=\sum_{\abs{\alpha}\leq k}\norm{D^\alpha f}_{L^2({\mathbb R}^d\to\mathbb{C}^m)}\qquad f\in H^k({\mathbb R}^d\to\mathbb{C}^m).\]
Note that it is not enough to require \(\abs{\alpha}=k\) as, for
example, the tempered distribution
\(1 \in {\mathcal S}'({\mathbb R}^d\to\mathbb{C})\) has a derivative
equal to zero however is not itself in
\(L^2({\mathbb R}^d\to\mathbb{C})\). Now, since, as we saw in (link),
the Fourier transform is an automorphism of
\({\mathcal S}'({\mathbb R}^d\to\mathbb{C}^m)\), by using property
(link) we deduce that
\[D^\alpha f\in L^2({\mathbb R}^d\to\mathbb{C}^m)\iff \mathcal{F}(D^\alpha f)=\abs{(2\pi i\xi )^{\alpha}} \hat{f}(\xi)\in L^2({\mathbb R}^d\to\mathbb{C}^m)\]
from which we deduce that \[\label{sobolevcondition}
    f\in H^k({\mathbb R}^d\to\mathbb{C}^m)\iff \sum_{\abs{\alpha}\leq k} \abs{(2\pi i\xi )^{\alpha}}\hat{f}(\xi)\sim_k \left\langle\xi\right\rangle^k\hat{f}\in L^2({\mathbb R}^d\to\mathbb{C}^m)\]
In fact, since the Fourier transform is a unitary transformation on
\(L^2({\mathbb R}^d\to\mathbb{C}^m)\), the same reasoning gives
\[\label{sobolevnorm}
    \norm{f}_{H^k({\mathbb R}^d\to\mathbb{C}^m)}\sim_k \norm{\left\langle\xi\right\rangle^k\hat{f}(\xi)}_{L^2({\mathbb R}^d\to\mathbb{C}^m)}\]
From the two equations above we deduce that if we define for a given
real number \(s\) (including negative numbers!) the \(s\)-th order
Sobolev space as
\[H^s({\mathbb R}^d\to\mathbb{C}^m):=\lbrace f\in L^2({\mathbb R}^d\to\mathbb{C}^m): \left\langle\xi\right\rangle^s\hat{f}(\xi)\in L^2({\mathbb R}^d\to\mathbb{C}^m)\rbrace\]
and give it the norm
\[\norm{f}_{H^k({\mathbb R}^d\to\mathbb{C}^m)}:= \norm{\left\langle\xi\right\rangle^s\hat{f}(\xi)}_{L^2({\mathbb R}^d\to\mathbb{C}^m)}\]
then our new definition is equivalent to the previous one (linkto3) when
\(s\) is a positive integer. We have thus found how to generalize the
concept of Sobolev space to all real orders and obtained a useful way of
characterizing them and giving a neat expression for their norm.
Nonetheless, it will always be useful to retain the first definition
based on derivatives, as it carries with it the motivation behind the
definition of Sobolev spaces. As was the case with tempered
distributions we can extend the concept of Sobolev space to periodic
domains by defining given an integer \(k\) the Sobolev space
\(H^k({\mathbb T}^d\to\mathbb{C}^m)\) as the space of square-integrable
\({\mathbb Z}^d\) periodic functions with distributional derivatives
themselves square integrable. Explicitly we define: \[\begin{gathered}
\label{sobolevdef2}
    H^k({\mathbb T}^d\to\mathbb{C}^m):=\\\lbrace f\in L^2({\mathbb T}^d\to\mathbb{C}^m): D^\alpha f\in L^2({\mathbb T}^d\to\mathbb{C}^m)\hookrightarrow{\mathcal S}'({\mathbb T}^d\to\mathbb{C}^m)\quad\forall\hspace{2pt}\abs{\alpha}\leq k\rbrace\end{gathered}\]
Using the same method as before, this time by Proposition 4 (link) and
equation of the correspondence between regularity and decay for the
Fourier transform of periodic functions (link), we deduce that
\[D^\alpha f\in H^k({\mathbb T}^d\to\mathbb{C}^m)\iff\widehat{D^\alpha f}(k)=\abs{k^\alpha}\hat{f}(k)\in l^2({\mathbb Z}^d\to\mathbb{C}^m)\]
which leads us as in the previous case to define for \(s\in{\mathbb R}\)
the more general Sobolev space
\[H^s({\mathbb T}^d\to\mathbb{C}^m):=\lbrace f\in L^2({\mathbb R}^d\to\mathbb{C}^m): \left\langle k\right\rangle^{s}\hat{f}(k)\in l^2({\mathbb Z}^d\to\mathbb{C}^m)\rbrace\]
and to give it the norm \[\label{sobolevgeneraldef}
    \norm{f}_{H^s({\mathbb T}^d\to\mathbb{C}^m)}:=\left(\sum_{k\in{\mathbb Z}^d}\left\langle k\right\rangle^{2s}\abs{\hat{f}(k)}^2\right)^{\frac{1}{2}}\]
where of course the definitions in (link) and (link) coincide for
\(s\in\mathbb{N}\). Note that, by the previous discussion, we have that
both in the euclidean and periodic case \[\begin{aligned}
\label{Sobolev derivatives embedding}
    f\in H^s({\mathbb R}^d\to\mathbb{C}^m)  & \iff  D^\alpha f\in H^{s-\abs{\alpha}}({\mathbb R}^d\to\mathbb{C}^m)\qquad\forall\abs{\alpha}\leq s  \\
    f\in H^s({\mathbb T}^d\to\mathbb{C}^m) & \iff  D^\alpha f\in H^{s-\abs{\alpha}}({\mathbb T}^d\to\mathbb{C}^m)\qquad\forall\abs{\alpha}\leq s\end{aligned}\]
One major advantage of working with the Sobolev spaces \(H^s\) is that,
differently to the classical space of smooth functions \(C^s\), they
form a Hilbert space with the inner product given by \[\begin{aligned}
    \left\langle f,g\right\rangle_{H^s({\mathbb R}^d\to\mathbb{C}^m)}  & :=\int_{{\mathbb R}}^d}\left\langle\xi \right\rangle^{2s}\hat{f}(\xi )\overline{\hat{g}(\xi )} d\xi \\
    \left\langle f,g\right\rangle_{H^s({\mathbb T}^d\to\mathbb{C}^m)} & :=\sum_{k\in {\mathbb Z}^d} \left\langle k \right\rangle^{2s}\hat{f}(k)\overline{\widehat{g}(k )}.\end{aligned}\]
This gives one access to all the power of functional analysis and is
invaluable in proofs. However, at the end of the day wishes to prove
that solutions with smooth initial data are themselves smooth in a
classical sense. This can be done by showing that the solution belongs
to a Sobolev space of high enough order together with the following two
results.

\textbf{Lemma 2} (\textbf{Continuity of Holder functions}). \emph{Given
\(f\in H^{s}({\mathbb T}^d\to\mathbb{C}^m)\) with \(s>d/2\). Then the
Fourier series of \(f\) is absolutely convergent and
\(f\in C({\mathbb T}^d\to\mathbb{C}^m)\) with the bound
\[\norm{f}_{L^\infty({\mathbb T}^d\to\mathbb{C}^m)}\lesssim_{d,s}\norm{f}_{H^s({\mathbb T}^d\to\mathbb{C}^m)}\]}

\emph{Proof.} The proof is an application of the Cauchy-Schwartz
inequality and
(\protect\hyperlink{HsisL1periodic}{{[}HsisL1periodic{]}}). We have that
\[\begin{gathered}
        \sum_{k\in{\mathbb Z}^d}\abs{\hat{f}(k)}=\sum_{k\in{\mathbb Z}^d} \left\langle k\right\rangle^{-s}\left\langle k\right\rangle^{s}\abs{\hat{f}(k)}\leq\left(\sum_{k\in{\mathbb Z}^d} \frac{\left\langle k\right\rangle^{-2s}}{2}\right)^{\frac{1}{2}}\left(\sum_{k\in{\mathbb Z}^d}\frac{\left\langle k\right\rangle^{2s}}{2}\abs{\hat{f}(k)}^2\right)^{\frac{1}{2}}\\
        \lesssim_{d,s}\norm{f}_{H^s(\mathbb{T}^d\to\mathbb{C}^m)}<\infty.
    \end{gathered}\] In consequence, the sum \[\label{Fourier sum}
        \sum_{k\in{\mathbb Z}^d}\hat{f}(k)e^{2\pi ik\cdot x}\] converges
absolutely (and uniformly). Since by Plancherel's Theorem, the above sum
also converges in \(L^2({\mathbb T}^d\to\mathbb{C}^m)\) to \(f\) we
deduce that \protect\hyperlink{Fourierux20sum}{{[}Fourier sum{]}}
converges almost everywhere to \(f\) (for example by taking a
subsequence of the above sum that converges almost everywhere to \(f\)).
Therefore
\[\norm{f}_{L^\infty({\mathbb T}^d\to\mathbb{C}^m)}=\norm{\sum_{k\in{\mathbb Z}^d}\hat{f}(k)e^{2\pi ik\cdot x}}_{L^\infty({\mathbb T}^d\to\mathbb{C}^m)}\]
which is
\[\leq\sum_{k\in{\mathbb Z}^d}\norm{\hat{f}(k)e^{2\pi ik\cdot x}}_{L^\infty({\mathbb T}^d\to\mathbb{C}^m)}=\sum_{k\in{\mathbb Z}^d}\abs{\hat{f}(k)}\lesssim_{d,s}\norm{f}_{H^s({\mathbb T}^d\to\mathbb{C}^m)}.\]
The continuity of \(f\) follows from the point-wise equality
\[\label{pointwise convergence Fourier sum}
        f(x)=\sum_{k\in{\mathbb Z}^d}\hat{f}(k)e^{2\pi ik\cdot x}=\int_{{\mathbb Z}^d} \hat{f}(k)e^{2\pi ik\cdot x} dk\]
together with the monotone convergence theorem applied to
\({\mathbb Z}^d\) with the counting measure \(dk\).~◻

As a corollary of this, we have the following two results

\leavevmode\vadjust pre{\hypertarget{characterizationux20smoothux20functionsux20inux20Sobolevux20space}{}}%
\textbf{Proposition 7} (\textbf{Sobolev embedding}). \emph{Let
\(f\in H^s({\mathbb T}^d\to\mathbb{C}^m)\) where \(s>\frac{d}{2}+k\).
Then \(f\in C^k({\mathbb T}^d\to\mathbb{C}^m)\).}

\emph{Proof.} By
(\protect\hyperlink{Sobolevux20derivativesux20embeddingux20periodic}{{[}Sobolev derivatives embedding periodic{]}}),
we may apply the previous proposition to deduce that \(D^\alpha f\) is
continuous for all \(\abs{\alpha}\leq k\). Therefore it suffices to show
that for \(\abs{\alpha}\leq k\) the distributional derivatives
\(D^\alpha f\) are also the classical derivatives of \(f\) which we
denote by \(f_\alpha\). By the hypothesis placed on \(f\), we have that
the series
\[\sum_{k\in{\mathbb Z}^d} (2\pi ik)^\alpha \hat{f}(k)e^{2\pi ik\cdot x}\]
is absolutely convergent (by Lemma
\protect\hyperlink{Sobolevacseriesperiodic}{{[}Sobolevacseriesperiodic{]}}),
and hence, we may commute the derivatives of \(f\) with the sum in its
Fourier series to deduce the point-wise equality
\[f_\alpha(x)=\sum_{k\in{\mathbb Z}^d} (2\pi ik)^\alpha \hat{f}(k)e^{2\pi ik\cdot x}.\]
Now, note that by using
\protect\hyperlink{derivativedistr}{{[}derivativedistr{]}} for the
Fourier coefficients of distributions we also have that the equality
\[D^\alpha f(x)=\sum_{k\in{\mathbb Z}^d} (2\pi ik)^\alpha \hat{f}(k)e^{2\pi ik\cdot x}\]
holds in \(L^2({\mathbb T}^d\to\mathbb{C}^m)\). From these last two
equalities, we deduce that \(f_\alpha=D^\alpha f\) almost everywhere
which concludes our proof.~◻

As before the previous results also have a euclidean analog whose proof
is identical in replacing all of the above sums over \({\mathbb Z}^d\)
with integrals over \({\mathbb R}^d\). Finally, we conclude this post
with a neat little trick. Given smooth \(f\) and some differential
operator \(\mathcal{L}=\sum_{\alpha} D^\alpha=p(D)\) we have that
\[D^\alpha f =\mathcal{F}^{-1}(\mathcal{F}\mathcal{L}f)=\mathcal{F}^{-1}(p_\mathcal{L}(2 \pi i\xi )\hat{f}(\xi ))\]
The term \(p_\mathcal{L}\) is called a Fourier multiplier and there is
no need to limit ourselves to polynomials. In fact, we may make the
general definition that for a function of two variables \(p\)
\[p(x,D)f:=\mathcal{F}^{-1}(p(x,\xi )\hat{f}(\xi )).\] This leads to the
definition of \emph{pseudo-differential operators}. A particular case is
that of fractional operators. We now show an example.

\textbf{Example 2}. \emph{Given smooth \(f\) we have that
\[\Delta f=\mathcal{F}^{-1}(-4 \pi^2 \abs{x_i}^2 \hat{f}(\xi )).\] As a
result, we define for all \(s\in {\mathbb R}\)
\[{(-\Delta)}^sf:=\mathcal{F}^{-1} ((4\pi^2 \left\langle\xi \right\rangle^2)^s\hat{f}(\xi )).\]}

This post is already getting a bit long (if you've stuck in till the end
I salute you), so we leave it off here with a neat little exercise (our
first of this blog)

\textbf{Exercise 1}. \emph{How should we define
\[T:=\left(1-\frac{\Delta}{4\pi ^2}  \right)^\frac{s}{2}\text{?}\] Once
you do so show that for all \(s,k \in {\mathbb R}\), the operator \(T\)
defines a linear bijective isometry
\[T: H^k({\mathbb R}^d\to\mathbb{C}^m)\to H^{k-s}({\mathbb R}^d\to\mathbb{C}^m).\]}

In our future post we will discuss the well posedness of second order
elliptic equations, and if we have time discuss the heat equation and
how the Fourier transform affords provides us with an elegant solution.

\end{document}
