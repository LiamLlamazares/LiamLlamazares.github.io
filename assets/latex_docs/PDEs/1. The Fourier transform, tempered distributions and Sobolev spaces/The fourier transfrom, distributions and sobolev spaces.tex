\documentclass[12pt]{article}
\special{papersize=3in,5in}
\usepackage[utf8]{inputenc}
%PACKAGES
\usepackage{CJKutf8}
\usepackage[colorlinks = true,
	linkcolor = blue,
	urlcolor  = black,
	citecolor = blue,
	anchorcolor = blue]{hyperref}
\usepackage[T1]{fontenc}
\makeatletter
\def\ps@pprintTitle{%
	\let\@oddhead\@empty
	\let\@evenhead\@empty
	\let\@oddfoot\@empty
	\let\@evenfoot\@oddfoot
}
\usepackage{amssymb,amsmath,physics,amsthm,xcolor,graphicx}
\usepackage[shortlabels]{enumitem}
\newtheorem{observation}{Observation}
\newtheorem{theorem}{Theorem}
\newtheorem{proposition}{Proposition}
\newtheorem{lemma}{Lemma}
\newtheorem{definition}{Definition}
\newtheorem{corollary}{Corollary}
\newtheorem{example}{Example}
\newtheorem{exercise}{Exercise}
\newcommand{\red}[1]{{\color{red}#1}}
\usepackage[colorlinks = true,
	linkcolor = blue,
	urlcolor  = black,
	citecolor = blue,
	anchorcolor = blue]{hyperref}
\usepackage{cleveref}
\bibliographystyle{elsarticle-num}
\newcommand{\A}{\mathbb{A}}\newcommand{\C}{\mathbb{C}}\newcommand{\E}{\mathbb{E}}\newcommand{\F}{\mathbb{F}}\newcommand{\K}{\mathbb{K}}\newcommand{\LL}{\mathbb{L}}\newcommand{\M}{\mathbb{M}}\newcommand{\N}{\mathbb{N}}\newcommand{\PP}{\mathbb{P}}\newcommand{\Q}{\mathbb{Q}}\newcommand{\R}{{\mathbb R}}\newcommand{\TT}zzzzz\newcommand{\Z}{{\mathbb Z}}
\newcommand{\Ww}{\mathcal{W}}\newcommand{\Aa}{\mathcal{A}}\newcommand{\Bb}{\mathcal{B}}\newcommand{\Cc}{\mathcal{C}}\newcommand{\Ee}{\mathcal{E}}\newcommand{\Ff}{\mathcal{F}}\newcommand{\Gg}{\mathcal{G}}\newcommand{\Hh}{\mathcal{H}}\newcommand{\Kk}{\mathcal{K}}\newcommand{\Ll}{\mathcal{L}}\newcommand{\Mm}{\mathcal{M}}\newcommand{\Nn}{\mathcal{N}}\newcommand{\Pp}{\mathcal{P}}\newcommand{\Qq}{\mathcal{Q}}\newcommand{\Rr}{{\mathcal R}}\newcommand{\Ss}{{\mathcal S}}\newcommand{\Tt}{{\mathcal T}}\newcommand{\Zz}{{\mathcal Z}}\newcommand{\Uu}{{\mathcal U}}
\newcommand{\W}{{\mathcal W}}
\newcommand\restr[2]{{\left.\kern-\nulldelimiterspace #1\vphantom{\big|} \right|_{#2}}}
\newcommand{\br}[1]{\left\langle#1\right\rangle}
\pagestyle{empty}
\setlength{\parindent}{0in}

\begin{document}
\begin{CJK*}{UTF8}{gbsn}
	\title{The Fourier transform, tempered distributions and Sobolev spaces}
	\author{Liam Llamazares}
	\date{10-14-2022}
	\maketitle
	\section*{ Three line summary }
	\begin{itemize}
		\item The \emph{Fourier transform} defines a linear isometry on the space of square-integrable complex-valued functions and has as inverse the \emph{inverse Fourier transform}(link). This allows us to write these functions as a superposition of harmonic functions.(link)
		\item Linear differential operators act as a simple polynomial multiplication on the frequency domain.(link) Smooth functions have Fourier transforms that decay quickly and vice-versa(link). This correspondence also allows us to introduce fractional differential operators and pseudo differential operators such as $\sqrt{-\Delta }$(link).
		\item The Sobolev spaces $H^s$ and tempered distributions  $\Ss'$ are complete spaces of functions to which we can extend differential operators.(link)
	\end{itemize}
	\section*{Introduction}
	Hi everyone, welcome to the first post in a new series focusing on partial differential equations (PDEs). In this post, we'll introduce some of the key analytical tools used to solve them. These tools will play a prominent role in the future posts of this series and in other series to come.  We'll begin by introducing some notation and necessary preliminaries on integration. Next, we will introduce the Fourier transform, which is the fundamental object of harmonic analysis and plays a fundamental role in partial differential equations (PDEs), probability theory, and number theory. In particular, we will show in future posts how many linear PDEs such as the heat equation can be solved using it.\\
	\\
	After this, we'll discuss the generalization of what it means to solve a PDE. To do so we will need to introduce the space of (tempered) distributions $\Ss'$ and Sobolev spaces $H^s$.	These two spaces will be crucial in allowing us to extend the concept of solution to PDE(link) to these spaces and use their completeness to prove the existence, uniqueness, and regularity of solutions.\\
	\\
	Much of the material here contained can be found in my master's thesis(link) on the Navier Stokes equations which in turn are based on Terence Tao's excellent notes(link).
\end{CJK*}
\subsection*{Notation}
We  write $L^p(\R^d)$ and $L^p(\R^d\to\C^m)$ to denote the space of equivalence classes of $p$ integrable functions from $\R^d$ to $\R$ and $\C^m$ respectively.\bigbreak
Given a function defined on a factor space $u:X\times Y\rightarrow Z$ we shall use $u(x)$ to stand for the slightly more cumbersome $u(x,\cdot ):Y\rightarrow Z$.
Given $\alpha=(\alpha_1,...,\alpha_d)\in\N^d$ we shall write as is standard \[x^\alpha:=x^{\alpha_1}\cdots  x^{\alpha_d};\quad D^\alpha:=\partial_1^{\alpha_1}\cdots\partial_1^{\alpha_d}\]
and whenever the expression $D^\alpha$ appears we will assume implicitly that $\alpha\in\N^d$.
Given  $x\in\R^d$ we will write $\abs{x}$ to denote its norm and the \emph{Japanese bracket} $\br{x}$ will signify
\[\br{x}:=(1+\abs{x}^2)^{1/2}.\]

Finally, we will employ the notation $f\lesssim g$ to mean that there exists some constant $C$ such that $f\leq Cg$. If the value of $C$ depends on some other parameter such as the dimension $d$ we shall make this explicit by writing $f\lesssim_d g$.\bigbreak
\subsection*{Some useful results on integration}
In PDEs, it is essential to be able to commute limits and differentials with both summation and integration. The main way of doing this is the dominated and monotone convergence theorem (link) together with the following two propositions which we state in full generality for functions valued in a Banach space $E$. Though for now, we will only need the case $E=\R$. On first reading, this can be assumed for commodity.
\begin{proposition}[Continuity of the integral]\label{commuting limits with integral}
	Let $(X,\mu )$ be a measure space, $T$ a first countable metric space, $(E,\norm{})$ a Banach space and
	\[f:T\times X\to E\]
	such that $f(t)$ is (Bochner(link)) integrable for all ${t}\in{T}$ and such that
	\[\norm{f(t)}\leq g\in L^1(X)\quad \forall{t}\in {T}.\]
	Then given $t_0\in T$ we have that
	\[\lim_{t\to t_0}\int_X f(t,x)\mu(dx) =\int_X\lim_{t\to t_0}f(t,x)\mu(dx) .\]
	In consequence, if $f(x)$ is continuous so is $\int_X f(t,x) \mu(dx)$.
\end{proposition}
\begin{proof}
	The proof is an application of the dominated convergence theorem(link) to the
	sequence of functions $f_n:=f(t_n)$ where $\{t_n\}_{n=1}^\infty$ is some sequence converging to $t$ and the fact that, in first countable spaces, sequential continuity(link) is equivalent to continuity.
\end{proof}
\begin{proposition}[{Differentiation under the integral sign}]\label{derivationundertheintsign}
	Let $X$ be a measure space,  $U$ be a convex open interval of $\R$, and $E$ a Banach space. Consider \[f:U\times X\to E\] such that:
	\begin{itemize}[a)]
		\item $ f(t)$ is measurable for every $t\in U$ and integrable for some $t_0\in U$.
		\item  For each $x\in X$ we have that $f(x)$ is differentiable and there exists an integrable function $g: X\to\R$ such that
		      \[\norm{\partial_{t}{f}(t,x)}\leq g(x)\quad \forall{(t,x)}\in {U\times E}.\]
		      Then it holds that
		      \[\partial_{t}{\int_X f(t)\mu(dx)}=\int_X \partial_{t}f(t) \mu(dx).\]

	\end{itemize}
\end{proposition}
\begin{proof}
	First, we show that the integral on the left-hand side of the equation above makes sense. Let $t\in U$ be any, then by the mean value inequality we have that
	\begin{equation*}
		\frac{f(t)-f(t_0)}{t-t_0}\leq \sup_{s \in U}\norm{\partial _t f(s)}\leq g \in L^1(X).
	\end{equation*}
	Thus, we may apply the previous proposition to commute the limit with the integral
	\begin{multline*}
		\lim_{t \to t_0}\frac{\int_{X} f(t)\mu (dx)-\int_{X}f(t_0) \mu (dx)}{t-t_0}  =\lim_{t \to t_0}\int_{X}    \frac{f(t)-f(t_0)}{t-t_0} \mu (dx) \\=\int_{X}\lim_{t \to t_0} \frac{f(t)-f(t_0)}{t-t_0} \mu (dx) =\int_{X}\partial _t f(t) \mu(dx).
	\end{multline*}
	Where in the first equality we also used the linearity of the Bochner integral (link) and in the last equality we used that by hypothesis $f(x)$ is everywhere differentiable. This concludes the proof.
\end{proof}
In the literature the usual hypothesis is that $\partial _t f$ is integrable everywhere (see for example \cite{FF} page 108), however as we have seen it is sufficient to require the integrability at a point.
\section{The Fourier transform on $\R^d$}
These preliminaries out of the way, we now get to our first main section. We assume the reader has encountered the Fourier transform before, and limit ourselves to giving a rundown of the theory and main ideas, taking the reader through a basic definition to an extension of it. To make the Fourier transform an isometry we will use the following convention.
\begin{definition}
	Given $f\in L^1(\R^d\to\C^m)$ the \emph{Fourier transform} of $f$ and the \emph{inverse Fourier transform of} $f$ are defined respectively by
	\[\hat{f}(\xi):=\int_{\R^d}f(x)e^{-2\pi i\xi\cdot  x} dx, \quad \check{f}(\xi):=\int_{\R^d}f(x)e^{2\pi i\xi\cdot  x} dx \]
\end{definition}
Other exponents such as $e^{-ix}, e^{-\pi i x}$ can also be used with identical results modulo constants. Let us introduce the notation $$\Ff_1 f:=\hat{f}, \Ff_1^*f$$ for the Fourier transform (on $L^1$) and its inverse.  The following proposition holds.
\begin{proposition}[\textbf{ Properties of $\Ff_1$}]\label{regularitydecayft}
	Given $f\in L^1(\R^d\to\C^m)$
	\begin{enumerate}
		\item The Fourier transform   $\Ff_1$ is a continuous linear operators $$\Ff_1:L^1(\R^d\to\C^m) \to L^\infty(\R^d\to\C^m)$$.
		\item If $x^\alpha f(x)\in L^1(\R^d\to\C^m)$ then:\quad \[D^\alpha \hat{f}(\xi)=(-2\pi i)^\abs{\alpha}\widehat{x^{\alpha} f}(\xi)\qquad\forall \xi\in\R^d.\]
		\item If $f$ is absolutely continuous in $x_j$ for almost every $x_1,...x_{j-1},x_{j+1},...,x_d$ then
		      then
		      \[\widehat{\partial x_j f}(\xi)=2\pi i\xi_j\widehat{ f}(\xi)\qquad\forall \xi\in\R^d.\]
	\end{enumerate}
\end{proposition}
\begin{proof}
	The proof of the first point is an application of the triangle inequality as
	\begin{equation*}
		\abs{\hat{f}(\xi )}=\abs{\int_{\R^d}f(x)e^{2\pi i x\cdot \xi } dx}\leq\int_{\R^d}\abs{f(x)}dx\quad\forall \xi \in \R^d.
	\end{equation*}
	and similarly for the inverse. The second point follows by differentiation under the integral(link). The third point can be proved by using integration by parts and the observation that every absolutely continuous function must go to zero at infinity.
\end{proof}
We note that the analogous holds for the inverse Fourier transform where it is only necessary to remove the minus sign in $2$ and a change of the sign in $3$. The proof can be completed
in an identical fashion or by noting that $\check{f}(\xi )=\hat{f}(-\xi )$.
\bigbreak
The importance of these last two properties is that they give us information on the Fourier transform of a function $f$ based on its regularity and decay. They can be stated informally as decay gives regularity and regularity gives decay (as the Fourier transform of any $f\in L^1(\R^d\to\R^d)$ decays to 0 at infinity by the Riemann-Lebesgue lemma(link)).\bigbreak
In general, the Fourier transform of an integrable function is not itself integrable. That is, $L^1(\R^d\to\C^m)$ is not closed under the Fourier transform. We now introduce a space that is closed under the Fourier transform; the Schwartz space, which can be thought of as the space of infinitely regular functions with infinite decay: \bigbreak
\begin{definition}\label{Schwartz space definition}
	The \emph{Schwartz space} $\mathcal{S}(\R^d\to\C^m)$ is:
	\[\mathcal{S}(\R^d\to\C^m):=\lbrace f\in\C^\infty(\R^d\to\C^m):x^\alpha D^\beta f\in L^\infty(\R^d\to\C^m)\quad\forall\hspace{2pt}\alpha,\beta\in \N^d\rbrace.\]
\end{definition}
By applying Proposition $1$ (link) we deduce that, given $f\in\mathcal{S}(\R^d\to\C^m)$ (linkme1)
\begin{align*}\label{rdft}
	\xi^\alpha D^\beta \hat{f}(\xi)   & =(-2\pi i)^{\abs{\beta}}\xi^{\alpha}\widehat{x^{\beta}f}=\frac{(-2\pi i)^{\abs{\beta}}}{(2\pi i)^{\abs{\alpha}}}\widehat{D^{\alpha}x^{\beta}f}\in L^\infty(\R^d\to\C^m) \\
	\xi^\alpha D^\beta \check{f}(\xi) & =\frac{(2\pi i)^{\abs{\beta}}}{(-2\pi i)^{\abs{\alpha}}}\widehat{D^{\alpha}x^{\beta}f}\in L^\infty(\R^d\to\C^m).
\end{align*}
Hence, the Fourier transform and the inverse Fourier transform restrict to endomorphisms of the Schwartz space which we will denote respectively by
\[\mathcal{F}_\mathcal{S},\mathcal{F}^*_\mathcal{S}:\mathcal{S}(\R^d\to\C^m)\to\mathcal{S}(\R^d\to\C^m).\]
The next item on the agenda is Plancherel's theorem which is proven via the following lemma, in which we shall use the notation\[\br{f,g}_{L^2(\R^d\to\C^m)}:=\int_{\R^d}f(x)\cdot  \overline{g(x)} dx\] for the inner product on $L^2(\R^d\to\C^m)$.
\begin{lemma}[\textbf{Plancherel for Schwartz functions}]\label{Planncherellemma}
	Given $f,g\in \mathcal{S}(\R^d\to\C^m)$ \bigbreak
	(i) $\br{\mathcal{F}_\mathcal{S}f,g}_{L^2(\R^d\to\C^m)}=\br{f,\mathcal{F}_\mathcal{S}^*g}_{L^2(\R^d\to\C^m)};\quad\br{\mathcal{F}_\mathcal{S}^*f,g}_{L^2(\R^d\to\C^m)}=\br{f,\mathcal{F}_\mathcal{S}g}_{L^2(\R^d\to\C^m)}$
	\bigbreak
	(ii) $\mathcal{F}_\mathcal{S}^*\mathcal{F}_\mathcal{S}f=\mathcal{F}_\mathcal{S}\mathcal{F}_\mathcal{S}^*f=f$
\end{lemma}
The first point can be proved by a direct application of Fubini. The second point is trickier and is the crux of why we call $\Ff^*$ the \emph{inverse} Fourier transform. The proof can be found in \cite{Tay} pages 222-226.
From $(i)$ and $(ii)$ we deduce immediately that, given a Schwartz function $f$,
\begin{equation}\label{Plancherel}
	\norm{f}_{L^2(\R^d\to\C^m)}=\norm{\mathcal{F}_\mathcal{S}f}_{L^2(\R^d\to\C^m)}=\norm{\mathcal{F}_\mathcal{S}^*f}_{L^2(\R^d\to\C^m)}
\end{equation}
That is, the restrictions of $\Ff_1$ and $\Ff^*_1$ \[\mathcal{F}_\mathcal{S},\mathcal{F}_\mathcal{S}^*:\mathcal{S}(\R^d\to\C^m)\to\mathcal{S}(\R^d\to\C^m)\]
are linear unitary operators which are the inverse the one to the other. We obtain the following result.\bigbreak
\begin{proposition}[\textbf{Plancherel's theorem}]\label{Plancherel's Theorem}\bigbreak $\mathcal{F}_\mathcal{S}$ and $\mathcal{F}_\mathcal{S}^*$ may be extended to unitary operators:
\[\mathcal{F},\mathcal{F}^*:L^2(\R^d\to\C^m)\to L^2(\R^d\to\C^m)\]
With $\Ff\Ff^*=\Ff^*\Ff=Id$.
\end {proposition}
\begin{proof}
	This is a immediate result of the density of $\mathcal{S}(\R^d\to\C^m)$ in the larger $L^2(\R^d\to\C^m)$ and the completeness of $L^2(\R^d\to\C^m)$ together with $(ii)$ and Plancherel (link). As, by a simple limiting argument, it is easy to see that continuous extensions of continuous linear operators preserve the norm and the inverses of the operators in question.
\end{proof}
It is reasonable to wonder if this extended Fourier transform $\Ff$ coincides with our initial definition when reasonable.
That is, whether given a function $f \in L^1(\R^d\to\C^m)\cap L^2(\R^d\to\C^m)$ we have that
\[\Ff f(\xi)=\Ff_1f(\xi)=\int_{\R^d}f(x)e^{-2\pi ix\cdot \xi} dx\qquad\forall \xi\in\R^d.\]
This indeed holds as can be seen by taking a sequence of functions $\lbrace f_n\rbrace_{n=1}^\infty\in\Ss(\R^d\to\C^m)$ converging to $f$ in  $L^1(\R^d\to\C^m)$ and in $L^2(\R^d\to\C^m)$.
As then $\Ff_1 f_n$ converges uniformly to $\Ff_1f$ by $1$ (point one prop 1 link) and in $L^2(\R^d\to\C^m)$ to $\Ff f$ by Proposition 2 (link), from where we deduce that as desired $\Ff_1f=\Ff f$. \bigbreak

\section{The Fourier transform of periodic functions}
We now extend our previous results to $\Z^d$ periodic functions, though, for any period, analogous results may be achieved. By identifying $\Z^d$ periodic functions with functions on the torus $\TT^d:=\R^d/\Z^d\equiv [0,1]^d$ we will work with functions $f:\TT^d\to\C^m.$ As we will see strikingly similar results are achieved in this setting
\begin{definition}
	Given $f\in L^1(\TT^d\to\C^m)$ we define the \emph{$k$-th Fourier coefficient} of $f$ as:
	\[\hat{f}(k):=\int_{\TT^d}f(x)e^{-2\pi ik\cdot x}dx.\]
\end{definition}
\noindent We thus obtain a function $\hat{f}$ on $\Z^d$  which we shall call the Fourier series of $f$ and a continuous linear function which we shall denote as in the euclidean case:
\[\Ff_1:L^1(\TT^d\to\C^m)\to l^\infty(\Z^d\to\C^m)\]
where $l^{\infty}(\Z^d\to\C^m)$ is the set of bounded sequences from $\Z^d$ to $\C^m$. As before we have the following result:
\begin{proposition}\label{regdecaypft}
	Given $f\in L^1(\TT^d\to\C^m)$ if $f$ is absolutely continuous in $x_j$ for almost every $x_1,...x_{j-1},x_{j+1},...,x_d$ then
	\[\widehat{\partial_{x_j}{f}}(k)=2\pi i k_j\widehat{f}(k)\qquad\forall k\in\Z^d.\]
\end{proposition}
In particular if $f\in C^\infty(\TT^d\to\C^m)$ we have that:
\begin{equation}\label{rgivesdpft}
	\widehat{D^\alpha f}(k)=(2\pi ik)^\alpha\hat{f}(k),
\end{equation}
and as a result we have that $\hat{f}$ is of rapid decrease (i.e. $\hat{f}$ decreases faster than the inverse of any polynomial). We thus have, as before, an induced map:
\[\Ff_{C^\infty}:C^\infty(\TT^d\to\C^m)\to s(\Z^d\to\C^m)\]
\[f\mapsto \hat{f}.\]
Where $s(\Z^d\to\C^m)$ are the sequences from $\Z^d$ to $\C^m$ that are of rapid decrease (this space now plays the role of the Schwartz space $\Ss(\R^d\to\C^m)$).
\bigbreak
Similarly to the non-periodic case we now define
\[\Ff^*_{C^\infty}:s(\Z^d\to\C^m)\to C^\infty(\TT^d\to\C^m);\quad a\mapsto\check{a}\]

with:
\[\check{a}(x):=\sum_{k\in\Z^d}a(k)e^{2\pi ik\cdot x}.\]
This is indeed smooth as, by the rapid decay of $a$, we may  differentiate under the integral sign(link) (we recall that sums are just integrals against discrete measures) to commute all derivatives with the above sum to obtain that
\begin{equation}\label{dgivesrpft}
	D^\alpha \check{a}(x)=\sum_{k\in\Z^d}(2\pi ik)^\alpha a(k)e^{2\pi ik\cdot x}\quad\forall{a}\in {s(\Z^d\to\C^m)}.
\end{equation}
It is now possible to prove as with the euclidean case that
\[\Ff_{C^\infty}\Ff^*_{C^\infty}=\Ff^*_{C^\infty}\Ff_{C^\infty}=Id,\]
and that analogously given $f$ smooth, and $a$ of rapid decrease
\[\br{\Ff_{C^\infty} f,a}_{l^2(\Z^d\to\C^m)}=\br{f,\Ff_{C^\infty}^*a}_{L^2(\R^d\to\C^m)};\quad\br{\Ff_{C^\infty}^* a,f}_{L^2(\R^d\to\C^m)}=\br{a,\Ff f}_{l^2(\Z^d\to\C^m)}.\]
See for example \cite{Tay} pages 197-206. We conclude that $\Ff_{C^\infty}$ are unitary linear functions and that hence:
\begin{proposition}[\textbf{Plancherel's (periodic) theorem}] \label{Plancherelperiodictheorem}
	$\mathcal{F}_{C^\infty}$ and $\mathcal{F}_{C^\infty}^*$ may be extended to unitary operators:
	\[\mathcal{F}:L^2(\TT^d\to\C^m)\to l^2(\TT^d\to\C^m);\quad \mathcal{F}^*:l^2(\Z^d\to\C^m)\to L^2(\R^d\to\C^m)\]
	with $\Ff\Ff^*=\Ff^*\Ff=Id.$
\end{proposition}
We note that, as for the Euclidean case, given $f\in L^2(\TT^d\to\C^m)\cap L^1(\TT^d\to\C^m)$ by an identical argument it holds that
\[\Ff f(k)=\Ff_1f(k)=\int_{\R^d}f(x)e^{-2\pi ik\cdot x}\]
and where now Plancherel's theorem gives that, for such $f$:
\begin{equation}\label{Plancerelpft}
	f(x)=\sum_{k\in\Z^d}\hat{f}(k)e^{2\pi ik\cdot x}.
\end{equation}

\section{Distributions and Sobolev Spaces}
Here we will quickly recall the concepts of tempered distributions and Sobolev spaces, which are concepts of utmost importance in the field of PDEs and Fourier analysis.
The idea is as follows, given a topological vector space $V$ we denote the dual of $V$  by \[V':=\lbrace w:E\to\C\hspace{2pt}: w \hspace{2pt}\text{continuous}\rbrace.\]
Furthermore, given $w\in V'$ and $u\in V$ we use the notation
\begin{equation*}
	(u,w):=w(u)
\end{equation*}
The reason for this is that $w(u)$ can often be interpreted as the inner product of $w$ against $u$ in a suitable Hilbert space. In our case, this Hilbert space will be $L^2(\R^d\to\C)$ and the above interpretation will allow us to extend differentiation and the Fourier transform by a technique called ``duality''.
\bigbreak  We will apply this where $V=\Ss(\R^d\to\C)$, first we need to introduce a topology on $E$.
In general, given a real vector space $V$ together with a countable family of semi-norms $\lbrace p_j\rbrace_{j=0}^\infty$ with the property that: given $x\neq 0$, there exists $j$ such that $p_j(x)\neq 0$. Then
\begin{equation}\label{seminormsgivemetric}
	d(x,y):=\sum_{j=0}^\infty 2^{-j}\frac{p_j(x-y)}{1+p_j(x-y)}\quad \forall{x,y}\in {E}
\end{equation}
is a translation invariant metric on $E$. In the case of the Schwartz space $\Ss(\R^d\to\C)$ we give it the topology induced by
\begin{equation}\label{seminornormsSchwartz}
	p_k(u):=\sum_{\abs{\alpha}\leq k} \sup_{x\in\R^d}\br{x}^k \abs{D^\alpha u(x)}.
\end{equation}
Though other families of semi-norms the reader may be familiar with such as
\[p_{k,\alpha}:=\sup_{x\in\R^d}\abs{x}^k \abs{D^\alpha u(x)};\quad\text{or}\quad p'_{k,\alpha}:=\sup_{x\in\R^d}(1+\abs{x})^k \abs{D^\alpha u(x)}\]
induce equivalent topologies. We note that with this metric $\Ss(\R^d\to\C)$ is complete.




For a quick proof based on the
fundamental theorem of calculus see for example \cite{Foll} page 237.
\begin{definition}
	The space of \emph{tempered distributions} is the dual space to $\Ss(\R^d\to\C)$ with the topology generated by $p_k)$. We write it $\Ss'(\R^d\to\C)$.
\end{definition}
One may verify that we have the inclusion
\begin{align}\label{lpisdistr}
	L^p(\R^d\to\C)\hookrightarrow\Ss'(\R^d\to\C);\quad f\mapsto T_f
\end{align}
where given $u\in \Ss(\R^d\to\C)$ we define \[T_f(u):=\br{u,f}:=\int_{\R^d} u\overline{f}.\]
Let us write $T^t$  for the transpose of a linear function $T$ and recall that the Fourier transform is an endomorphism of the Schwartz space. Then, given two Schwartz functions $u,v$ we have already seen that, by a simple application of Fubini,
\[T_{\Ff v}(u)=\br{u,\Ff v}=\br{\Ff^{-1} u,v}=(\Ff^{-1}u,T_v)\]
and integration by parts gives
\[T_{D^\alpha v}(u)=\br{u,D^\alpha v}=(-1)^\abs{\alpha}\br{D^\alpha u,v}=((-1)^\abs{\alpha}D^\alpha u,T_v)\quad\forall{\alpha}\in {\N^d}.\]

\bigbreak
This gives us a way of extending the Fourier transform and differentiation to the space of tempered distributions.
\begin{definition}
	Given $w\in\mathcal{S}'(\R^d\to\C)$ and $\alpha\in\N^d$ we define the \emph{(distributional) Fourier transform} of $w$ by
	\[\Ff w:= w\circ \Ff^{-1}\]
	and the \emph{(weak) $\alpha$'th} derivative of $w$ by
	\[D^\alpha w:= w\circ((-1)^\abs{\alpha}D^\alpha).\]

\end{definition}
The method we used above to extend $\Ff, D^\alpha$ is called the \emph{duality method} and appears very frequently. Another way of writing the above definition is:
\begin{equation*}
	(u, \Ff w):=(\Ff^{-1} u, w);\quad(u,D^\alpha w):=(-1)^\abs{\alpha}(D^\alpha u,w)
\end{equation*}
Due to our previous discussion, we have that with this definition
\begin{equation}\label{ftlpdistr}
	\Ff T_u=T_{\Ff u};\quad D^\alpha T_{u}=T_{D^\alpha u}\quad\forall u\in \Ss(\R^d).
\end{equation}
Just as one would desire.\bigbreak

Two other operations that are permitted on $\Ss(\R^d\to\C)$ are multiplication by functions of polynomial growth $p$ and the application of the inverse Fourier transform which we shall, as for $L^2$ functions, denote by $\Ff^{-1}$. Both definitions are once again being given by duality.
\begin{equation*}
	(u,pw):=(\overline{p}u,w);\quad (u,\Ff^{-1}w):=(\Ff u,w) .
\end{equation*}

Before ending our discussion of (scalar) tempered distributions we comment on some generalizations. We first note that the previous discussion works equivalently for vector-valued distributions, i.e. elements of the dual to $\mathcal{S}(\R^d\to\C^m)$, which we shall denote by\[\mathcal{S}'(\R^d\to\C^m)\]
where the only change is that the inclusion of integrable functions is now given by
\begin{align}\label{lpisvectordistr}
	L^p(\R^d\to\C^m)\hookrightarrow\Ss'(\R^d\to\C^m);\quad f\mapsto T_f
\end{align}
with $T_f$ defined by (linkme2)
\[T_f(u):=\int_{\R^d} u\cdot \bar{f}\quad\forall{u}\in {\Ss(\R^d\to\C^m)}.\]
In both cases we have that, by duality, due to the formulas derived previously (link1), given a tempered distribution $w$ and $\alpha\in \N^d$
\begin{align}\label{derivativedistronperiodic}
	D^\alpha\Ff w =(-2\pi i)^{\abs{\alpha}}\Ff x^\alpha w;\quad
	\Ff D^{\alpha} w= (2\pi i)^{\abs{\alpha}}x^\alpha\Ff w.
\end{align}
As we have observed before (link) multiplication by functions of polynomial growth is a well-defined operation on $\Ss(\R^d\to\C^m)$ so the above expressions are also well defined. A quick verification also shows that, since Plancherel's theorem holds for all Schwartz functions,
\begin{equation}\label{planchereldistr}
	\Ff^{-1}\Ff w=\Ff\Ff^{-1}w=w.
\end{equation}

Finally, in addition to ``changing the image'' of our distributions, we may also ``change the domain'' by considering, for example, periodic tempered distributions. Where now, as we saw in the section on the Fourier transform, $C^\infty(\TT^d\to\C^m)$ takes the place of the Schwartz space and where we place on $C^\infty(\TT^d\to\C^m)$ the topology defined by the countable family of semi-norms:
\[q_k(u):=\sum_{\abs{\alpha}\leq k}\sup_{x\in\TT^d}\abs{D^\alpha u}\]
and denote its dual by $\Ss'(\TT^d\to\C^m).$ Note that, as the domain is bounded, multiplication by polynomials to ensure rapid decrease is redundant.
By defining as is natural the Fourier series of a periodic distribution $w$ by the sequence (which can be shown to be of polynomial growth)
\begin{equation}\label{fouriercoeffperiodicdist}
	\hat{w}(k):=(e^{-2\pi ikx},w)\quad k\in\Z^d
\end{equation}
and its $\alpha$-th distributional derivative by
\[(u,D^\alpha w):=(-1)^\abs{\alpha}(D^\alpha u,w)\]
we derive formulas analogous to the ones seen in the section on the Fourier transform for ``periodic" distributions as well. Namely:
\begin{equation}\label{derivativedistr}
	w=\sum_{k\in\Z^d}\hat{w}(k)e^{2\pi i k x};\quad\widehat{D^\alpha w}(k)=(2\pi ik)^\alpha\hat{w}(k).
\end{equation}

To prove it all we have to do is apply Plancherel on $u$ and move terms around via duality,
\begin{multline*}
	(u,w)=(\sum_{k\in \Z^d}\hat{u}(k)e^{2\pi ik\cdot x},w )=\sum_{k\in \Z^d}  \hat{u}(k)\hat{w}(k)\\=\sum_{k\in \Z^d}  \left(\int_{\TT^d} u(x)e^{-2\pi i\omega\cdot  x}dx\right)\hat{w}(k)=\int_{\TT^d} u(x)\left(\sum_{k\in \Z^d} \hat{w}(k) e^{-2\pi i\omega\cdot  x}\right)dx.
\end{multline*}
This proves the first part and the second can be proved directly by considering the relevant definitions.\\
\\
To sum up what we have seen, due to the natural inclusion of integrable functions in the space of tempered distributions and the analogous inclusion in the periodic case, the notion of Fourier transform and differentiation extends to the larger space of tempered distributions. This allows us to manipulate rough functions (periodic or non-periodic) as if they had Fourier transforms and were smooth. As we shall see, this will prove of great use when obtaining ``distributional solutions'' to some PDEs. We now give the general method by which this achieved
\begin{definition}
	Consider a mapping
	\[P:A\subset\Ss(\R^d\to\C^m)\to \Ss(\R^d\to\C^m)\]
	that extends to
	\[P:S\subset \Ss'(\R^d\to\C^n)\to \Ss'(\R^d\to\C^n)\]
	where $S$  is some subset of ${\Ss'}(\R^d\to\C^m)$ containing $\Ss^\infty(\R^d\to\C^m)$. Then, given $f \in \Ss'(\R^d\to \C^n)$ we say that a \emph{distributional solution} to $Pw=f$ is any tempered distribution $w\in S$ verifying
	$Pw=f$.
\end{definition}
In the above definition, $P$ typically defines a linear or non-linear differential equation. Note that the above definition may be extended without any difficulty to the case of (periodic) distributional solutions in the case where $P: C^\infty(\TT^d\to\C^m)\to C^\infty(\TT^d\to\R^n)$.
\begin{example}
	Set $P=\Delta $, as we have seen previously (link) $\Delta$ extends to $S:=\Ss'(\R^d\to\C^m)$ with
	\begin{equation*}
		\Ff(\Delta \omega)=-4\pi \abs{\xi }^2\Ff w .
	\end{equation*}
	As a result we deduce that for any $f \in \Ss'(\R^d\to \C^n)$ the equation $Pw=f$ has as it's unique solution
	\begin{equation*}
		w=\Ff^{-1}\left(\frac{-f}{4\pi \abs{\xi }^2 }\right).
	\end{equation*}
\end{example}


\bigbreak
\section{Sobolev spaces}
Sobolev spaces form a particular case of tempered distributions that we interpret as being smooth and integrable up to sufficient orders (linkme3).
\begin{definition}\label{sobolevdef}
	Given $k\in\N^+$ we define the \emph{Sobolev space} $H^k(\R^d\to\C^m)$ as:
	\begin{multline*}
		H^k(\R^d\to\C^m):=\\ \lbrace f\in L^2(\R^d\to\C^m): D^\alpha f\in L^2(\R^d\to\C^m)\hookrightarrow\Ss '(\R^d\to\C^m)\quad\forall\hspace{2pt}\abs{\alpha}\leq k\rbrace.
	\end{multline*}
	Where,  we consider $L^2(\R^d\to\C^m)$ as a subspace of $\Ss '(\R^d\to\C^m)$ (see linkto2).\end{definition} We may interpret the Sobolev space $H^k(\R^d\to\C^m)$ as the space of $k$ times differentiable functions in $L^2(\R^d\to\C^m)$ and we give it the norm:
\[\norm{f}_{H^k(\R^d\to\C^m)}:=\sum_{\abs{\alpha}\leq k}\norm{D^\alpha f}_{L^2(\R^d\to\C^m)}\qquad f\in H^k(\R^d\to\C^m).\]
Note that it is not enough to require $\abs{\alpha}=k$ as, for example, the tempered distribution $1 \in \Ss'(\R^d\to\C)$ has a derivative equal to zero however is not itself in $L^2(\R^d\to\C)$. Now, since, as we saw in (link), the Fourier transform is an automorphism of $\Ss'(\R^d\to\C^m)$, by using property (link)  we deduce that
\[D^\alpha f\in L^2(\R^d\to\C^m)\iff \Ff (D^\alpha f)=\abs{(2\pi i\xi )^{\alpha}} \hat{f}(\xi)\in L^2(\R^d\to\C^m)\]
from which we deduce that
\begin{equation}\label{sobolevcondition}
	f\in H^k(\R^d\to\C^m)\iff \sum_{\abs{\alpha}\leq k} \abs{(2\pi i\xi )^{\alpha}}\hat{f}(\xi)\sim_k \br{\xi}^k\hat{f}\in L^2(\R^d\to\C^m)
\end{equation}
In fact, since the Fourier transform is a unitary transformation on $L^2(\R^d\to\C^m)$, the same reasoning gives
\begin{equation}\label{sobolevnorm}
	\norm{f}_{H^k(\R^d\to\C^m)}\sim_k \norm{\br{\xi}^k\hat{f}(\xi)}_{L^2(\R^d\to\C^m)}
\end{equation}
From the two equations above we deduce that if we define for a given real number $s$ (including negative numbers!) the $s$-th order Sobolev space as
\[H^s(\R^d\to\C^m):=\lbrace f\in L^2(\R^d\to\C^m): \br{\xi}^s\hat{f}(\xi)\in L^2(\R^d\to\C^m)\rbrace\]
and give it the norm
\[\norm{f}_{H^k(\R^d\to\C^m)}:= \norm{\br{\xi}^s\hat{f}(\xi)}_{L^2(\R^d\to\C^m)}\]
then our new definition is equivalent to the previous one (linkto3)  when $s$ is a positive integer. We have thus found how to generalize the concept of Sobolev space to all real orders and obtained a useful way of characterizing them and giving a neat expression for their norm. Nonetheless, it will always be useful to retain the first definition based on derivatives, as it carries with it the motivation behind the definition of Sobolev spaces.\bigbreak
As was the case with tempered distributions we can extend the concept of Sobolev space to periodic domains by defining given an integer $k$ the Sobolev space $H^k(\TT^d\to\C^m)$ as the space of square-integrable $\Z^d$ periodic functions with distributional derivatives themselves square integrable. Explicitly we define:
\begin{multline}\label{sobolevdef2}
	H^k(\TT^d\to\C^m):=\\\lbrace f\in L^2(\TT^d\to\C^m): D^\alpha f\in L^2(\TT^d\to\C^m)\hookrightarrow\Ss '(\TT^d\to\C^m)\quad\forall\hspace{2pt}\abs{\alpha}\leq k\rbrace
\end{multline}
Using the same method as before, this time by Proposition 4 (link) and equation of the correspondence between regularity and decay for the Fourier transform of periodic functions (link), we deduce that
\[D^\alpha f\in H^k(\TT^d\to\C^m)\iff\widehat{D^\alpha f}(k)=\abs{k^\alpha}\hat{f}(k)\in l^2(\Z^d\to\C^m)\]
which leads us as in the previous case to define for $s\in\R$ the more general Sobolev space
\[H^s(\TT^d\to\C^m):=\lbrace f\in L^2(\R^d\to\C^m): \br{k}^{s}\hat{f}(k)\in l^2(\Z^d\to\C^m)\rbrace\]
and to give it the norm
\begin{equation}\label{sobolevgeneraldef}
	\norm{f}_{H^s(\TT^d\to\C^m)}:=\left(\sum_{k\in\Z^d}\br{k}^{2s}\abs{\hat{f}(k)}^2\right)^{\frac{1}{2}}
\end{equation}
where of course the definitions in (link) and (link) coincide for $s\in\N$.
Note that, by the previous discussion, we have that both in the euclidean and periodic case
\begin{align}\label{Sobolev derivatives embedding}
	f\in H^s(\R^d\to\C^m)  & \iff  D^\alpha f\in H^{s-\abs{\alpha}}(\R^d\to\C^m)\qquad\forall\abs{\alpha}\leq s  \\
	f\in H^s(\TT^d\to\C^m) & \iff  D^\alpha f\in H^{s-\abs{\alpha}}(\TT^d\to\C^m)\qquad\forall\abs{\alpha}\leq s
\end{align}
One major advantage of working with the Sobolev spaces $H^s$ is that, differently to the classical space of smooth functions $C^s$, they form a Hilbert space with the inner product given by
\begin{align*}
	\br{f,g}_{H^s(\R^d\to\C^m)}  & :=\int_{\R^d}\br{\xi }^{2s}\hat{f}(\xi )\overline{\hat{g}(\xi )} d\xi \\
	\br{f,g}_{H^s(\TT^d\to\C^m)} & :=\sum_{k\in \Z^d} \br{k }^{2s}\hat{f}(k)\overline{\widehat{g}(k )}.
\end{align*}
This gives one access to all the power of functional analysis and is invaluable in proofs. However,  at the end of the day wishes to prove that solutions with smooth initial data are themselves smooth in a classical sense. This can be done by showing that the solution belongs to a Sobolev space of high enough order together with the following two results.
\begin{lemma}[\textbf{Continuity of Holder functions}]
	Given $f\in H^{s}(\TT^d\to\C^m)$ with $s>d/2$. Then the Fourier series of $f$ is absolutely convergent and $f\in C(\TT^d\to\C^m)$ with the bound
	\[\norm{f}_{L^\infty(\TT^d\to\C^m)}\lesssim_{d,s}\norm{f}_{H^s(\TT^d\to\C^m)}\]
\end{lemma}
\begin{proof}
	The proof is an application of the Cauchy-Schwartz inequality and (\ref{HsisL1periodic}). We have that
	\begin{multline*}
		\sum_{k\in\Z^d}\abs{\hat{f}(k)}=\sum_{k\in\Z^d} \br{k}^{-s}\br{k}^{s}\abs{\hat{f}(k)}\leq\left(\sum_{k\in\Z^d} \frac{\br{k}^{-2s}}{2}\right)^{\frac{1}{2}}\left(\sum_{k\in\Z^d}\frac{\br{k}^{2s}}{2}\abs{\hat{f}(k)}^2\right)^{\frac{1}{2}}\\
		\lesssim_{d,s}\norm{f}_{H^s(\mathbb{T}^d\to\C^m)}<\infty.
	\end{multline*}
	In consequence, the sum
	\begin{equation}\label{Fourier sum}
		\sum_{k\in\Z^d}\hat{f}(k)e^{2\pi ik\cdot x}
	\end{equation}
	converges absolutely (and uniformly). Since by Plancherel's Theorem, the above sum also converges in $L^2(\TT^d\to\C^m)$ to $f$ we deduce that \eqref{Fourier sum} converges almost everywhere to $f$ (for example by taking a subsequence of the above sum that converges almost everywhere to $f$). Therefore
	\[\norm{f}_{L^\infty(\TT^d\to\C^m)}=\norm{\sum_{k\in\Z^d}\hat{f}(k)e^{2\pi ik\cdot x}}_{L^\infty(\TT^d\to\C^m)}\]
	which is
	\[\leq\sum_{k\in\Z^d}\norm{\hat{f}(k)e^{2\pi ik\cdot x}}_{L^\infty(\TT^d\to\C^m)}=\sum_{k\in\Z^d}\abs{\hat{f}(k)}\lesssim_{d,s}\norm{f}_{H^s(\TT^d\to\C^m)}.\]
	The continuity of $f$ follows from the point-wise equality
	\begin{equation}\label{pointwise convergence Fourier sum}
		f(x)=\sum_{k\in\Z^d}\hat{f}(k)e^{2\pi ik\cdot x}=\int_{\Z^d} \hat{f}(k)e^{2\pi ik\cdot x} dk
	\end{equation}
	together with the monotone convergence theorem applied to $\Z^d$ with the counting measure $dk$.
\end{proof}
As a corollary of this, we have the following two results
\begin{proposition}[\textbf{Sobolev embedding}]\label{characterization smooth functions in Sobolev space}
	Let $f\in H^s(\TT^d\to\C^m)$ where $s>\frac{d}{2}+k$. Then   $f\in C^k(\TT^d\to\C^m)$.
\end{proposition}
\begin{proof}
	By (\ref{Sobolev derivatives embedding periodic}),
	we may apply the previous proposition to deduce that $D^\alpha f$ is continuous for all $\abs{\alpha}\leq k$. Therefore it suffices to show that for $\abs{\alpha}\leq k$ the distributional derivatives $D^\alpha f$ are also the classical derivatives of $f$ which we denote by $f_\alpha$.\bigbreak
	By the hypothesis placed on $f$, we have that the series
	\[\sum_{k\in\Z^d} (2\pi ik)^\alpha \hat{f}(k)e^{2\pi ik\cdot x} \]
	is absolutely convergent (by Lemma \ref{Sobolevacseriesperiodic}), and hence, we may commute the derivatives of $f$ with the sum in its Fourier series to deduce the point-wise equality
	\[f_\alpha(x)=\sum_{k\in\Z^d} (2\pi ik)^\alpha \hat{f}(k)e^{2\pi ik\cdot x}.\]
	Now, note that by using \eqref{derivativedistr} for the Fourier coefficients of distributions we also have that the equality
	\[D^\alpha f(x)=\sum_{k\in\Z^d} (2\pi ik)^\alpha \hat{f}(k)e^{2\pi ik\cdot x}\]
	holds in $ L^2(\TT^d\to\C^m)$. From these last two equalities, we deduce that $f_\alpha=D^\alpha f$ almost everywhere which concludes our proof.
\end{proof}
As before the previous results also have a euclidean analog whose proof is identical in replacing all of the above sums over $\Z^d$ with integrals over $\R^d$.
Finally, we conclude this post with a neat little trick. Given smooth $f$ and some differential operator $\Ll =\sum_{\alpha} D^\alpha=p(D)$ we have that
\begin{equation*}
	D^\alpha f =\Ff^{-1}(\Ff  \Ll f)=\Ff^{-1}(p_\Ll(2 \pi i\xi )\hat{f}(\xi ))
\end{equation*}
The term $p_\Ll$ is called a Fourier multiplier and there is no need to limit ourselves to polynomials. In fact, we may make the general definition that for a function of two variables $p$
\begin{equation*}
	p(x,D)f:=\Ff^{-1}(p(x,\xi )\hat{f}(\xi )).
\end{equation*}
This leads to the definition of \emph{pseudo-differential operators}. A particular case is that of fractional operators. We now show an example.
\begin{example}
	Given smooth $f$ we have that
	\begin{equation*}
		\Delta f=\Ff^{-1}(-4 \pi^2 \abs{x_i}^2 \hat{f}(\xi )).
	\end{equation*}
	As a result, we define for all $s\in \R$
	\begin{equation*}
		{(-\Delta)}^sf:=\Ff^{-1} ((4\pi^2 \br{\xi }^2)^s\hat{f}(\xi )).
	\end{equation*}
\end{example}
This post is already getting a bit long (if you've stuck in till the end I salute you), so we leave it off here with a neat little exercise (our first of this blog)
\begin{exercise}
	How should we define $$T:=\left(1-\frac{\Delta}{4\pi ^2}  \right)^\frac{s}{2}\text{?} $$ Once you do so show that for all $s,k \in \R$, the operator $T$ defines a linear bijective isometry
	\begin{equation*}
		T: H^k(\R^d\to\C^m)\to H^{k-s}(\R^d\to\C^m).
	\end{equation*}
\end{exercise}
In our future post we will discuss the well posedness of second order elliptic equations, and if we have time discuss the heat equation and how the Fourier transform affords provides us with an elegant solution.



\bibliography{biblio.bib}
\end{document}
