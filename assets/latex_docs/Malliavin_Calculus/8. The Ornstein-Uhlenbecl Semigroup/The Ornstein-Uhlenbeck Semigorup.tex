\documentclass[12pt]{article}
\special{papersize=3in,5in}
\usepackage[utf8]{inputenc}
%PACKAGES
\usepackage{CJKutf8}
\usepackage[colorlinks = true,
	linkcolor = blue,
	urlcolor  = black,
	citecolor = blue,
	anchorcolor = blue]{hyperref}
\usepackage[T1]{fontenc}
\makeatletter
\def\ps@pprintTitle{%
	\let\@oddhead\@empty
	\let\@evenhead\@empty
	\let\@oddfoot\@empty
	\let\@evenfoot\@oddfoot
}
\usepackage{amssymb,amsmath,physics,amsthm,xcolor,graphicx}
\usepackage[shortlabels]{enumitem}
\newtheorem{observation}{Observation}
\newtheorem{theorem}{Theorem}
\newtheorem{proposition}{Proposition}
\newtheorem{lemma}{Lemma}
\newtheorem{definition}{Definition}
\newtheorem{corollary}{Corollary}
\newcommand{\red}[1]{{\color{red}#1}}
\usepackage[colorlinks = true,
	linkcolor = blue,
	urlcolor  = black,
	citecolor = blue,
	anchorcolor = blue]{hyperref}
\usepackage{cleveref}
\bibliographystyle{elsarticle-num}
\newcommand{\A}{\mathbb{A}}\newcommand{\C}{\mathbb{C}}\newcommand{\E}{\mathbb{E}}\newcommand{\F}{\mathbb{F}}\newcommand{\K}{\mathbb{K}}\newcommand{\LL}{\mathbb{L}}\newcommand{\M}{\mathbb{M}}\newcommand{\N}{\mathbb{N}}\newcommand{\PP}{\mathbb{P}}\newcommand{\Q}{\mathbb{Q}}\newcommand{\R}{{\mathbb R}}\newcommand{\T}{{\mathbb T}}\newcommand{\Z}{{\mathbb Z}}
\newcommand{\Aa}{\mathcal{A}}\newcommand{\Bb}{\mathcal{B}}\newcommand{\Cc}{\mathcal{C}}\newcommand{\Ee}{\mathcal{E}}\newcommand{\Ff}{\mathcal{F}}\newcommand{\Gg}{\mathcal{G}}\newcommand{\Hh}{\mathcal{H}}\newcommand{\Kk}{\mathcal{K}}\newcommand{\Ll}{\mathcal{L}}\newcommand{\Mm}{\mathcal{M}}\newcommand{\Nn}{\mathcal{N}}\newcommand{\Pp}{\mathcal{P}}\newcommand{\Qq}{\mathcal{Q}}\newcommand{\Rr}{{\mathcal R}}\newcommand{\Ss}zzzzz\newcommand{\Tt}{{\mathcal T}}\newcommand{\Zz}{{\mathcal Z}}\newcommand{\Uu}{{\mathcal U}}\newcommand{\Ww}{{\mathcal W}}

\newcommand\restr[2]{{\left.\kern-\nulldelimiterspace #1\vphantom{\big|} \right|_{#2}}}
\newcommand{\br}[1]{\left\langle#1\right\rangle}
\newcommand{\md}{\,\mmd}
\newcommand{\wt}[1]{\widetilde{#1}}
\newcommand{\ol}[1]{\overline{#1}}
\newcommand{\wh}[1]{\widehat{#1}}
\pagestyle{empty}
\setlength{\parindent}{0in}

\begin{document}
\begin{CJK*}{UTF8}{gbsn}
	\title{The Ornstein-Uhlenbeck Semigroup}
	\author{Liam Llamazares}
	\date{date}
	\maketitle
	\section{ Three line summary}
	\begin{itemize}
		\item There is a natural extension of the Laplacian to the Wiener space.
		\item The generator of the Laplacian is the Ornstein-Uhlenbeck semigroup.
		\item The Ornstein-Uhlenbeck semigroup in finite dimensions is the generator of the Ornstein-Uhlenbeck process, from which it derives its name.	\end{itemize}
	\section{The Laplacian of a random variable}
	First, we give some finite-dimensional motivation. Suppose that $f\in C_c^\infty(\R^d\to\R^d)$ and $g\in C_c^\infty(\R^d)$. Then an integration by parts shows that the adjoint of the gradient
	in $L^2(\R^d)$ is minus the divergence. That is,
	\begin{equation*}
		\int_{\R^d}f(x) \cdot \nabla g(x) dx=-\int_{\R^d}\nabla\cdot  f(x) \nabla g(x) dx.
	\end{equation*}
	Then, we define the Laplacian as minus the adjoint of the gradient $\nabla$ composed with the gradient
	\begin{equation*}
		\Delta := -\nabla^*\circ \nabla .
	\end{equation*}
	Which gives the familiar $$\Delta_{\R^d} =\nabla\cdot \nabla=\partial_1^2+\ldots\partial _d^2$$.

	Of course, this is all well and good when the domain of $f,g$ is a finite-dimensional space. Otherwise, there is no Lebesgue measure.	We now move to what is our base case in our series of blog posts and consider a probability space $(\Omega,\mathbb{P},\mathcal{F}_t)$ where $\Ff_t$ is generated by a Wiener process $W_t$. Then, as we have seen previously (link)
	the Skorohod integral $\delta $ is the adjoint of the Malliavin derivative $D$ so we would like to define
	\begin{equation*}
		\Delta  := -\delta \circ D.
	\end{equation*}
	On what kind of random variables can we define this? Well let us take $X=\sum_{n=0}^{\infty}  I_n(f_n)$ with a rapidly decaying chaos expansion, then
	\begin{equation*}
		\Delta X=-\delta (DX)=-\delta \left(\sum_{n=1}^\infty nI_{n-1}(f_n(\cdot ,t))\right)=-\sum_{n=1}^\infty nI_{n}(f_n).
	\end{equation*}
	All we require for this expression to make sense is that- the right-hand side is in $L^2(\Omega)$. That is, by Ito's $n$-th isometry (link), that
	\begin{equation*}
		\sum_{n=0}^\infty n^2 \norm{f_n}_{L^2(I_n)}< \infty.
	\end{equation*}
	Is this a space we've dealt with before? Well if we recall our old spaces $\mathbb{D}^{k,p}$  (link). Then we have that
	\begin{multline*}
		\int_{I^2}\norm{D_{t,s}X}_{L^2(\Omega)}^2 ds d t=\int_{I^2}\norm{\sum_{n=2}^\infty n(n-1)I_{n-2}(f_n(\cdot ,s,t))}_{L^2(\Omega)}^2\\=\int_{I^2}\sum_{n=2}^\infty n^2(n-1)^2(n-2)!\norm{f_n(\cdot ,s,t)}_{L^2(I_{n-2})}^2=\sum_{n=2}^\infty n(n-1)n!\norm{f_n(\cdot ,s,t)}_{L^2(I_n)}^2.
	\end{multline*}
	Where analogous calculations go through if we have more derivatives to get the terms $n(n-1)\cdots (n-(k-1))$. This shows that
	\begin{equation*}
		\mathbb{D}^{k,p}:=\left\{X\in L^2(\Omega):\quad \norm{X}_{\mathbb{D}^{k,2}}=\sum_{n=0}^\infty n^kn! \norm{f_n}_{L^p(I_n)}< \infty\right\} .
	\end{equation*}
	Thus, the domain of $\Delta $ is exactly  $\mathbb{D}^{2,2}$. This is quite pleasing as, as we have observed earlier, the spaces $\mathbb{D}^{k,p}$ mimic the Sobolev spaces $W^{k,p}$, when $p=2$ this resemblance is quite strong
	as we have that the norm on  $H^k:=W^{k,2}$ is
	\begin{equation*}
		\norm{f}_{H^k\R^d)}=\int_{\R^d}\br{\xi }^k \hat{f}(\xi )^2d\xi .
	\end{equation*}
	Which is formally equal to the one just derived for $\mathbb{D}^{k,2}.$ It is very interesting to observe that, directly from the definition, we obtain a basis of eigenvalues of $\Delta $. Let us define
	\begin{equation*}
		H_n:=\{X\in L^2(\Omega): X=I_n(f_n),\quad \text{for some } f_n \in L^2_S(I^n)  \} .
	\end{equation*}
	That is, $H_n$ are the random variables that only have the  $n$-th term in their chaos expansion to be non-zero. Then by the chaos expansion theorem, we know that
	\begin{equation*}
		L^2(\Omega)=\overline{\oplus_{n=0}^\infty H_n}.
	\end{equation*}
	And by construction of the Laplacian, $\Delta e_n=n e_n$ for every $e_n \in H_n$. In fact, by the uniqueness of the chaos expansion, the elements of $H_n$ for some  $n \in \N$ are the unique eigenvectors of $\Delta .$
	\section{The Ornstein-Uhlenbeck semigroup}
	As it turns out, $\Delta $ defines a semigroup
	\begin{definition}
		The \emph{Ornstein-Uhlenbeck semigroup} is the family of operators $\Phi(t):L^2(\Omega)\to L^2(\Omega)$
		\begin{equation*}
			\Phi(t)X:=\sum_{n=0}^{\infty}  e^{-nt}I_n(f_n),  \quad\forall t\in I.
		\end{equation*}

	\end{definition}
	The term $e^{-nt}$ is quite reminiscent of the semigroup for the heat equation
	\begin{equation*}
		e^{t\Delta }u_0:=\int_{\R^d}e^{-4 \pi^2 \xi^2t}\widehat{u_0}(\xi ) d\xi,
	\end{equation*}
	and will cause an analogous smoothing effect by making the terms in the chaos expansion to decrease	faster. To see that  $\Phi$ defines a semigroup first note that, by the linearity of the iterated integrals,
	\begin{equation*}
		\Phi(t)X:=\sum_{n=0}^{\infty}  I_n(e^{-nt}f_n).
	\end{equation*}
	So as a result
	\begin{equation*}
		\Phi(t+s)X=\sum_{n=0}^{\infty}  e^{-nt}I_n(e^{-ns}f_n)=\sum_{n=0}^{\infty}  \Phi(t)\Phi(s)X.
	\end{equation*}
	Which shows that $\Phi(t+s)=\Phi(t)\circ \Phi(s)$. Finally, note that
	\begin{equation*}
		\frac{\Phi(t)X-X}{t}=\sum_{n=0}^{\infty} \left(\frac{e^{-nt}-1}{nt} \right)nI_n(f_n)\to -\sum_{n=0}^{\infty}  nI_n(f_n)=\Delta X \in L^2(\Omega)  .
	\end{equation*}
	Where the commutation under the integral sign (with the counting measure) is justified as $(e^{-nt}-1)/(nt)$ is uniformly bounded in $n$.
	There's an explicit formula for $\Phi(t)$.
	\begin{proposition}[Mehler's formula]
		Let $(\Omega,\Ff_t,\gamma  )$ be the Wiener space, then
		\begin{equation*}
			\Phi(t)X(\omega)=\int_{\Omega}X\left(e^{-t}\omega+\sqrt{1-e^{-2t}}\eta\right) \gamma  (\eta)\in L^2(\Omega).
		\end{equation*}
	\end{proposition}
	The  proof is technical and can be found in  Nualart's book \cite{nualart2018introduction} on  page 74. Let us try to understand the formula and also the reason for the name of the semigroup. We consider as at the beginning of this post the finite-dimensional case but now with some Gaussian measure $\mu $
	\begin{equation*}
		\mu (A):=\int_{A}e^{-\frac{\norm{x}^2}{2} } dx.
	\end{equation*}
	Then, integration by parts shows that
	\begin{align*}
		\int_{\R^d}f(x) \cdot \nabla g(x) d\mu(x) & =-\int_{\R^d}\nabla\cdot  \left(e^{-\frac{\norm{x}^2}{2} }f(x)\right) \nabla g(x) dx \\&=\int_{\R^d}(x\cdot f(x)-\nabla\cdot f(x)) d\mu (x).
	\end{align*}
	That is, the adjoint of the gradient in $L^2(\R^d,\mu )$ is $x\cdot -\nabla\cdot $. Notice that we get the extra term that corresponds to multiplication by $x\cdot $.As a result, the Laplacian on $L^2(\R^d, \mu )$ is given by
	\begin{equation*}
		\Delta_\mu =\nabla\cdot \nabla-x\cdot \nabla .
	\end{equation*}
	Furthermore, by Itô's formula, $\Delta_\mu $ is the generator of the SDE
	\begin{equation*}
		dX(t)=-X(t)d t+ \sqrt{2}dW(t)
	\end{equation*}
	Let us write $X_x$ for the solution to the above SDE with initial data  $x \in \R^d$
	That is, if we define $$P_tX(x):=E[\varphi(X(t))],$$ then
	\begin{equation*}
		\partial _tP_tX(x)=\Delta_\mu P_tX(x) .
	\end{equation*}
	The process $X$ that solves the SDE above is known as the \emph{Ornstein-Uhlenbeck process} and, by the theory of linear SDEs, is given by
	\begin{equation*}
		X_x(t)=e^{-t}x+\sqrt{2} \int_{0}^te^{s-t} dW(t).
	\end{equation*}
	Since
	\begin{equation*}
		\sqrt{2} \int_{0}^te^{s-t} dW(t)\sim \sqrt{2} e^{-t}\Nn\left(0,\norm{e^\cdot }^2_{L^2([0,t])}\right)=\sqrt{1-e^{-2t}}\Nn(0,1)
	\end{equation*}
	We deduce that for each fixed $t$ we can find a measure $\gamma   \sim \Nn(0,1)$ with
	\begin{equation*}
		X(t)=e^{-t}x+\sqrt{1-e^{-2t}}\gamma .
	\end{equation*}
	We then get that
	\begin{equation*}
		P_t \varphi(x)=\E\left[\varphi\left(e^{-t}x+\sqrt{1-e^{-2t}}\gamma  \right)\right]
	\end{equation*}
	And by taking $\varphi=Id$ we recover Mehler's formula. This correspondence is expanded on in chapter $7$ of Hairer's notes \cite{hairer2009introduction}.
\end{CJK*}



\bibliography{biblio.bib}
\end{document}
