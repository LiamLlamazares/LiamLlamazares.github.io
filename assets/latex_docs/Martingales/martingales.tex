\documentclass[12pt]{article}
\special{papersize=3in,5in}
\usepackage[utf8]{inputenc}
%PACKAGES
\usepackage[colorlinks = true,
	linkcolor = blue,
	urlcolor  = black,
	citecolor = blue,
	anchorcolor = blue]{hyperref}
\usepackage[T1]{fontenc}
\makeatletter
\def\ps@pprintTitle{%
	\let\@oddhead\@empty
	\let\@evenhead\@empty
	\let\@oddfoot\@empty
	\let\@evenfoot\@oddfoot
}
\usepackage{amssymb,amsmath,physics,amsthm,xcolor,graphicx}
\usepackage[shortlabels]{enumitem}
\newtheorem{observation}{Observation}
\newtheorem{theorem}{Theorem}
\newtheorem{proposition}{Proposition}
\newtheorem{lemma}{Lemma}
\newtheorem{definition}{Definition}
\newtheorem{corollary}{Corollary}
\newcommand{\red}[1]{{\color{red}#1}}
\usepackage[colorlinks = true,
	linkcolor = blue,
	urlcolor  = black,
	citecolor = blue,
	anchorcolor = blue]{hyperref}
\usepackage{cleveref}
\bibliographystyle{elsarticle-num}
\newcommand{\A}{\mathbb{A}}\newcommand{\C}{\mathbb{C}}\newcommand{\E}{\mathbb{E}}\newcommand{\F}{\mathbb{F}}\newcommand{\K}{\mathbb{K}}\newcommand{\LL}{\mathbb{L}}\newcommand{\M}{\mathbb{M}}\newcommand{\N}{\mathbb{N}}\newcommand{\PP}{\mathbb{P}}\newcommand{\Q}{\mathbb{Q}}\newcommand{\R}zzzzz\newcommand{\T}{{\mathbb T}}\newcommand{\Z}{{\mathbb Z}}
\newcommand{\Aa}{\mathcal{A}}\newcommand{\Bb}{\mathcal{B}}\newcommand{\Cc}{\mathcal{C}}\newcommand{\Ee}{\mathcal{E}}\newcommand{\Ff}{\mathcal{F}}\newcommand{\Gg}{\mathcal{G}}\newcommand{\Hh}{\mathcal{H}}\newcommand{\Kk}{\mathcal{K}}\newcommand{\Ll}{\mathcal{L}}\newcommand{\Mm}{\mathcal{M}}\newcommand{\Nn}{\mathcal{N}}\newcommand{\Pp}{\mathcal{P}}\newcommand{\Qq}{\mathcal{Q}}\newcommand{\Rr}{{\mathcal R}}\newcommand{\Tt}{{\mathcal T}}\newcommand{\Zz}{{\mathcal Z}}\newcommand{\Uu}{{\mathcal U}}
\newcommand\restr[2]{{\left.\kern-\nulldelimiterspace #1\vphantom{\big|} \right|_{#2}}}
\newcommand{\br}[1]{\left\langle#1\right\rangle}
\pagestyle{empty}
\setlength{\parindent}{0in}
\begin{document}
\title{Martingales in Banach spaces}
\author{Liam Llamazares}
\date{23-05-2022}
\maketitle
\section{Three line summary}
\begin{itemize}
	\item Conditional expectations exist in a natural way for simple functions, by taking extensions they also exist for integrable functions to a Banach space $L^1(\Omega\to E)$.
	\item Using conditional expectations we can define what a martingale is just like in the real case.
	\item The space of continuous $p$-integrable martingales is a Banach space.
\end{itemize}
\section{Why should I care?}
Banach valued martingales form the basis of SPDEs. This is because analogously to Itô integration of real-valued processes. Integrating against a Wiener process valued in a Banach space the same will produce a square integrable continuous martingale.
\section{Conditional expectation}
In graduate-level probability courses, given a $\sigma-$algebra $\Gg$ one shows that by applying Radon-Nikodyn's theorem, for any real-valued random variable $X\in L^1(\Omega\to R)$ there exists a conditional expectation $\E_{\Gg}[X]$ verifying that
\begin{equation*}
	\int_A \E_{\Gg}[X]=\int_A X,\quad \forall A\in\Gg.
\end{equation*}
Of course, now that we have created an integral for integral random variables to a Banach space $L^1(\Omega\to X)$ we would like to see whether such a conditional expectation also exists for these functions. If we are given a simple function
\begin{equation*}
	X=\sum_{k=1}^n x_k 1_{A_k}, \quad x_k\in E, A_k\in\Gg.
\end{equation*}
It is a simple calculation to show that, since $1_{A_k}$ are real-valued and thus $\E_{\Gg}[1_{A_k}]$ are well defined, then
\begin{equation*}
	\E_{\Gg}[X]=\sum_{k=1}^n x_k \E_{\Gg}[1_{A_k}],
\end{equation*}
verifies the desired formula. Furthermore, we have that $\mathbb{E}_\mathcal{F}$ is a linear, and pointwise continuous operator with
\begin{equation*}
	\left\|\E_{\Gg}[X]\right\|\leq \sum_{k=1}^n \|x_k\| \E_{\Gg}[1_{A_k}]=\E_{\Gg}\left[\sum_{k=1}^n \|x_k\|1_{A_k}\right]=\E_{\Gg}\left[\|X\|\right].
\end{equation*}
This allows us to show the following
\begin{theorem}[Existence and uniqueness of conditional expectation]
	Let $X\in L^1(\Omega\to E)$ for some Banach space $E$. Then $X$ has a conditional expectation satisfying
	\begin{equation*}
		\left\|\E_{\Gg}[X]\right\|\leq\E_{\Gg}\left[\|X\|\right].
	\end{equation*}
\end{theorem}
\begin{proof}
	We have already proved the above inequality for simple processes. By the previous post, \cite{Bochner} we can take $X_n$ converging to $X$ in $L^1(\Omega\to E)$ to obtain that
	\begin{multline*}
		\left\|\E_{\Gg}[X_n-X_m]\right\|\leq\E_{\Gg}\left[\|X_n-X_m\|\right]\\\implies \E[\left\|\E_{\Gg}[X_n]-\E_{\Gg}[X_m]\right\|]\leq \E\left[\|X_n-X_m\|\right]\to 0
	\end{multline*}
	As a result, $\E_{\Gg}[X_n]$ is a Cauchy sequence in $L^1(\Omega\to E)$ and converges to some function $Y$, passing to the limit in the defining equation for the conditional expectation shows that $Z=\mathbb{E}_\mathcal{G}[X]$. Finally, to prove uniqueness we have that if both $Z_1,Z_2$ satisfy
	\begin{equation*}
		\int_A Z_1=\int_A X=\int_A Z_2,\quad \forall A\in\Gg.
	\end{equation*}
	Then using the linearity of the integral we obtain that $w(Z_1)=w(Z_2)$ for all linear function $w$, so $Z_1=Z_2$.
\end{proof}
\section{Martingales}
Okay, so we leveraged some inequalities to prove the existence of a conditional expectation. This done, the following definition mimicking the real case is quite natural
\begin{definition}
	Let $\{M(t)\}_{t\in I}$, be a stochastic process on $(\Omega, \mathcal{F}, \mathbb{P})$ with a filtration $\{\mathcal{F}_{t}\}_{t \in I}$. The process $M$ is called an $\mathcal{F}_{t}$-martingale, if:
	\begin{enumerate}
		\item $M(t)\in L^1(\Omega\to E)$ for all $t\in I$
		\item $M(t):\mathcal{F}_{t} \to \mathcal{B}(E)$ for all $t\in I$,
		\item $\E_{\mathcal{F}_{s}}\left[M(t)\right]=M(s)$ for all $s \leq t$.
	\end{enumerate}
\end{definition}
The concept of submartingale is  defined by replacing the equality in 3. with a $\geq$. Let us abbreviate $\E_{\mathcal{F}_t}$ by $\E_t$. Then, as in the real case, we have the following.
\begin{lemma}[Norm is submartingale]\label{norm is submart}
	Let $M(t)$ be a martingale, then $\norm{M(t)}$ is a martingale
\end{lemma}
\begin{proof}
	We recall that, by the Hahn Banach theorem, it holds for any metric space that given $y\in E$
	\begin{equation*}
		\norm{z}=\sup_{\ell\in E^*:\|\ell\|=1}\ell(z)
	\end{equation*}
	As a result, by the linearity of the integral and abbreviating the supremum to just $\sup_\ell$,
	\begin{multline*}
		\|M(s)\|=\|\E_{s}[M(t)]\|= \sup_\ell \ell\left(\E_{s}[M(t)]\right)=\sup_\ell\|{\E_{s}\left[\ell(M(t))\right]\|}\\
		\leq \E_{s}\left[\sup_\ell\ell(M(t))\right]=\E_{s}\left[\|M(t)\|\right]
	\end{multline*}
\end{proof}
Let us recall the following result for real-valued martingales
\begin{lemma}[Doob's maximal Martingale inequality]
	Let $\{X_k\}_{k=1}^\infty$ be a real-valued sub-martingale. Then it holds that
	\begin{equation*}
		\norm{\max_{k\in\{1,...,n\}}X_k}_{L^p(\Omega)}\leq \frac{p}{p-1}\norm{X_n}_{L^p(\Omega)}
	\end{equation*}
	As a consequence, if $X_t,t\in[0,T]$ is left (or right) continuous then
	\begin{equation*}
		\norm{\max_{t\in[0,T]}X_k}_{L^p(\Omega)}\leq \frac{p}{p-1}\norm{X_T}_{L^p(\Omega)}.
	\end{equation*}
\end{lemma}
The idea of the above result is that, since $X_k$ is a submartingale, $X_k\lesssim X_{k+1}\lesssim...\lesssim X_n$. Getting from the continuous to the discrete case is possible by using the continuity of $X$ and approximating it on some finer and finer mesh $t_0,...,t_n$. This said, applying Doob's maximal martingale inequality together with the Lemma \ref{norm is submart} gives that
\begin{theorem}[Maximal Inequality]\label{doobs Hilbert} Let $p>1$ and let $E$ be a separable Banach space.
	If $M(t)$, is a right-continuous $E$-valued $\mathcal{F}_{t}$-martingale, then
	\begin{multline*}
		\left(E\left(\sup _{t \in[0, T]}\|M(t)\|^{p}\right)\right)^{\frac{1}{p}} \leq \frac{p}{p-1} \sup _{t \in[0, T]}\left(E\left(\|M(t)\|^{p}\right)\right)^{\frac{1}{p}} \\
		=\frac{p}{p-1}\left(E\left(\|M(T)\|^{p}\right)\right)^{\frac{1}{p}}
	\end{multline*}
\end{theorem}
\begin{proof}
	This follows by using that $\norm{M(t)}$ is a sub-martingale and Doob's maximal inequality.
\end{proof}
Doob's inequality is essentially an equality between different function norms we can place on the space of continuous Martingales and will provide a very powerful tool later on.
\begin{corollary}
	Let $M$ be a (left or right) continuous martingale to a separable Banach space $E$. Then the following are equivalent
	\begin{itemize}
		\item $M\in \hat{L}^\infty([0,T]\to \hat{L}^2(\Omega\to E))$
		\item  $M\in \hat{L}^2(\Omega\to \hat{L}^\infty([0,T]\to E))$
		\item $\mathbb{E}[\norm{M(T)}^2]<\infty$
	\end{itemize}
	Where we recall from the previous post that $\hat{L}^p$ symbolizes that $M$ may not be separately valued and only have an integrable norm. That said, the same reasoning shows that the above result also holds for the integrable $L^p$ spaces.
\end{corollary}
A useful space of Martingales is as follows
\begin{definition}
	Let $M(t)$ be a $E$ valued martingale with index set $I=[0,T]$, then we define
	\begin{equation*}
		\mathcal{M}_T^2(E):=\left\{\text{continuous martingales } M:\mathbb{E}[\norm{M(T)}^2]<\infty\right\}
	\end{equation*}
	and give it the norm
	\begin{equation*}
		\norm{M}_{\mathcal{M}_T^2(E)}:=\mathbb{E}[\norm{M(T)}^2].
	\end{equation*}
\end{definition}
By Theorem \ref{doobs Hilbert} we have that
\begin{equation*}
	\mathcal{M}_T^2(E)\subset \hat{L}^\infty([0,T]\to \hat{L}^2(\Omega\to E))\cap \hat{L}^2(\Omega\to \hat{L}^\infty([0,T]\to E)).
\end{equation*}
and that any of the norms of these spaces is equivalent to the one set on $\mathcal{M}_T^2(E)$. This is useful in the following result
\begin{proposition}
	Let $E$ be a separable Banach space, then $\mathcal{M}_T^2(E)$ is a Banach space.
\end{proposition}
\begin{proof}
	By the previous observation  and the completeness of the $\hat{L}^p$ spaces proved in the previous post, $\mathcal{M}_T^2(E)$ is a subspace of a Hilbert space. As a result, it is sufficient to show that it is closed. Let $M_n$ converge to $M$. Then, by the equivalence of the norms we have that $M_n(t)\to M(t)\in \hat{L}^1(\Omega\to E)\subset \hat{L}^2(\Omega\to E)$ so that for all $A\in\mathcal{F}_s$
	\begin{equation*}
		\int_A M(s)d\mathbb{P}=\lim_{n\to\infty}\int_A M_n(s)d\mathbb{P}=\lim_{n\to\infty}\int_A M_n(t)d\mathbb{P}=\int_A M(t)d\mathbb{P}.
	\end{equation*}
	This shows that $M$ is a martingale. Furthermore, as was seen in the previous post, there exists a subsequence $M_{n_k}$ such that
	\begin{equation*}
		\lim_{n\to\infty}{M_{n_k}}(\cdot,\omega)=M(\cdot,\omega)\in \hat{L}^\infty([0,T]\to E)\quad a.e.\quad \omega\in\Omega
	\end{equation*}
	Since $M_{n_k}(\cdot,\omega)$ are continuous and continuity is preserved by uniform limits this proves that $M$ is continuous almost everywhere. This concludes the proof.
\end{proof}
In future installments, we will prove that a Banach valued Wiener process belongs to this space and use it to define the stochastic integral that leads to the construction of SPDEs.
\begin{proposition}
	Let $W(t)$ be a $E$ valued $\Sigma$-Wiener process with respect to a filtration $\mathcal{F}_t$. Then $W(t)\in \mathcal{M}_T^2(E)$.
\end{proposition}
\begin{proof}
	It is a martingale as it is adapted and, given $A\in\Gg_S$ and $u\in E$, by the linearity of the integral and Independence of $W(t)-W(s)$ with $\Gg_s$
	\begin{multline*}
		\br{\int_A W(t)-W(s)d\PP,u}=\int_A \br{W(t)-W(s),u}d\PP\\=
		\PP(A)\E[\br{W(t)-W(s),u}]=0
	\end{multline*}
	As a result
	\begin{equation*}
		\int_A W(t)d\PP=\int_A W(s)d\PP=0\quad \forall A\in\Gg_s\implies \E_s[W(t)]=W(s).
	\end{equation*}
	Finally, we have that $\E[W(t)^2]=t<\infty$ for all $t$ and $W$ is continuous by construction. This concludes the proof.
\end{proof}

\bibliography{biblio.bib}
\end{document}
