\documentclass[12pt]{article}
\special{papersize=3in,5in}
\usepackage[utf8]{inputenc}
%PACKAGES
\usepackage{CJKutf8}
\usepackage[colorlinks = true,
	linkcolor = blue,
	urlcolor  = black,
	citecolor = blue,
	anchorcolor = blue]{hyperref}
\usepackage[T1]{fontenc}
\makeatletter
\def\ps@pprintTitle{%
	\let\@oddhead\@empty
	\let\@evenhead\@empty
	\let\@oddfoot\@empty
	\let\@evenfoot\@oddfoot
}
\usepackage{amssymb,amsmath,physics,amsthm,xcolor,graphicx}
\usepackage[shortlabels]{enumitem}
\newtheorem{observation}{Observation}
\newtheorem{theorem}{Theorem}
\newtheorem{proposition}{Proposition}
\newtheorem{lemma}{Lemma}
\newtheorem{definition}{Definition}
\newtheorem{corollary}{Corollary}
\newcommand{\red}[1]{{\color{red}#1}}
\usepackage[colorlinks = true,
	linkcolor = blue,
	urlcolor  = black,
	citecolor = blue,
	anchorcolor = blue]{hyperref}
\usepackage{cleveref}
\bibliographystyle{elsarticle-num}
\newcommand{\A}{\mathbb{A}}\newcommand{\C}{\mathbb{C}}\newcommand{\E}{\mathbb{E}}\newcommand{\F}{\mathbb{F}}\newcommand{\K}{\mathbb{K}}\newcommand{\LL}{\mathbb{L}}\newcommand{\M}{\mathbb{M}}\newcommand{\N}{\mathbb{N}}\newcommand{\PP}{\mathbb{P}}\newcommand{\Q}{\mathbb{Q}}\newcommand{\R}{{\mathbb R}}\newcommand{\T}zzzzz\newcommand{\Z}{{\mathbb Z}}
\newcommand{\Aa}{\mathcal{A}}\newcommand{\Bb}{\mathcal{B}}\newcommand{\Cc}{\mathcal{C}}\newcommand{\Ee}{\mathcal{E}}\newcommand{\Ff}{\mathcal{F}}\newcommand{\Gg}{\mathcal{G}}\newcommand{\Hh}{\mathcal{H}}\newcommand{\Kk}{\mathcal{K}}\newcommand{\Ll}{\mathcal{L}}\newcommand{\Mm}{\mathcal{M}}\newcommand{\Nn}{\mathcal{N}}\newcommand{\Pp}{\mathcal{P}}\newcommand{\Qq}{\mathcal{Q}}\newcommand{\Rr}{{\mathcal R}}\newcommand{\Tt}{{\mathcal T}}\newcommand{\Zz}{{\mathcal Z}}\newcommand{\Uu}{{\mathcal U}}
\newcommand\restr[2]{{\left.\kern-\nulldelimiterspace #1\vphantom{\big|} \right|_{#2}}}
\newcommand{\br}[1]{\left\langle#1\right\rangle}
\pagestyle{empty}
\setlength{\parindent}{0in}

\begin{document}
\begin{CJK*}{UTF8}{gbsn}
	\title{Approximating random Gaussian fields via SPDEs}
	\author{Liam Llamazares}
	\date{07-06-2022}
	\maketitle
	\section{ Three line summary}
	\begin{itemize}
		\item
	\end{itemize}
	\section{Why should I care?}
	Gaussian fields very used, very expensive, gotta do something, more details below.
	\section{Notation}
	We will consider a probability space $(\Omega,\Ff,\mathbb{P})$ a Banach space $E$ and a Gaussian random variable  $X:\Omega\to E$
	\section{Introduction}
	In any practical application, a Gaussian field on $\R^d$ can only be observed at a finite amount of points $s_1,\ldots,s_N\in \R^d$. The first thing that comes to mind is to construct a Gaussian field $u\sim \Nn(\mu ,\Sigma)$ where $\mu =(\mu(s_j))_{j=1}^N \in \R^n$ and $\Sigma=(C(s_i,s_j))_{i,j=1}^N \in \R^{N\times N}$ are the sample mean and covariance respectively.	 However, working directly with this expression is computationally unfeasible in all but the simplest cases. This is due to the fact that in to calculate the density of $u$ it is necessary to invert the  covariance matrix $\Sigma $ which has a cost of $O(N^3)$.
	\section{The theoretical framework}
	Essentially, what we need to do is approximate $u$ by something which is much more computationally tractable. To that end, let us consider a Gaussian random variable $u:\Omega\to H$ where $H$ is some Hilbert space (typically one could consider for example $E=L^2(\R^d)$). Our goal will be to construct a sequence of finite dimensional subspaces $H_m\subset H$ and approximations $u_m:\Omega\to H_m$ to our original variable $u$. Let $H_m$ have orthonormal basis $\phi_j$, then
	\begin{equation*}
		u_m=\sum_{n=0}^{m} \omega_n\phi_n;\quad \omega=(\br{u_m,\phi_1},\ldots,\br{u_m,\phi_n}).
	\end{equation*}
	And it can be shown (essentially by definition of Gaussian variable) that $\omega$ is an $m$-dimensional Gaussian variable. One way of carrying out such an approximation is to exploit the Karhunen-Loève expansion
	\begin{equation*}
		u=\sum_{n=0}^{\infty} \lambda _n^\frac{1}{2}\beta _n\phi_n.
	\end{equation*}
	Where $(\lambda _n,\phi_n)$ are eigenpairs of the covariance matrix of $u$ and $\beta _n$ are $1$-dimensional iid normal variables. And then write
	\begin{equation*}
		u_m:=\sum_{n=0}^{m}\lambda _n^\frac{1}{2}\beta _n\phi_n.
	\end{equation*}
	This was done for example in Dashti 2011 (cite Dan aat some point as well.)
\end{CJK*}



\bibliography{biblio.bib}
\end{document}
