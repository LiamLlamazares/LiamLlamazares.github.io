\documentclass[12pt]{article}
\special{papersize=3in,5in}
\usepackage[utf8]{inputenc}
%PACKAGES
\usepackage{CJKutf8}
\usepackage[colorlinks = true,
	linkcolor = blue,
	urlcolor  = black,
	citecolor = blue,
	anchorcolor = blue]{hyperref}
\usepackage[T1]{fontenc}
\makeatletter
\def\ps@pprintTitle{%
	\let\@oddhead\@empty
	\let\@evenhead\@empty
	\let\@oddfoot\@empty
	\let\@evenfoot\@oddfoot
}
\usepackage{amssymb,amsmath,physics,amsthm,xcolor,graphicx}
\usepackage[shortlabels]{enumitem}
\newtheorem{observation}{Observation}
\newtheorem{theorem}{Theorem}
\newtheorem{proposition}{Proposition}
\newtheorem{lemma}{Lemma}
\newtheorem{definition}{Definition}
\newtheorem{corollary}{Corollary}
\newcommand{\red}[1]{{\color{red}#1}}
\usepackage[colorlinks = true,
	linkcolor = blue,
	urlcolor  = black,
	citecolor = blue,
	anchorcolor = blue]{hyperref}
\usepackage{cleveref}
\bibliographystyle{elsarticle-num}
\newcommand{\A}{\mathbb{A}}\newcommand{\C}{\mathbb{C}}\newcommand{\E}{\mathbb{E}}\newcommand{\F}{\mathbb{F}}\newcommand{\K}{\mathbb{K}}\newcommand{\LL}{\mathbb{L}}\newcommand{\M}{\mathbb{M}}\newcommand{\N}{\mathbb{N}}\newcommand{\PP}{\mathbb{P}}\newcommand{\Q}{\mathbb{Q}}\newcommand{\R}{{\mathbb R}}\newcommand{\T}{{\mathbb T}}\newcommand{\Z}{{\mathbb Z}}
\newcommand{\Aa}{\mathcal{A}}\newcommand{\Bb}{\mathcal{B}}\newcommand{\Cc}{\mathcal{C}}\newcommand{\Ee}{\mathcal{E}}\newcommand{\Ff}{\mathcal{F}}\newcommand{\Gg}{\mathcal{G}}\newcommand{\Hh}{\mathcal{H}}\newcommand{\Kk}{\mathcal{K}}\newcommand{\Ll}{\mathcal{L}}\newcommand{\Mm}{\mathcal{M}}\newcommand{\Nn}{\mathcal{N}}\newcommand{\Pp}{\mathcal{P}}\newcommand{\Qq}{\mathcal{Q}}\newcommand{\Rr}{{\mathcal R}}\newcommand{\Tt}{{\mathcal T}}\newcommand{\Zz}{{\mathcal Z}}\newcommand{\Uu}{{\mathcal U}}
\newcommand\restr[2]{{\left.\kern-\nulldelimiterspace #1\vphantom{\big|} \right|_{#2}}}
\newcommand{\br}[1]{\left\langle#1\right\rangle}
\pagestyle{empty}
\setlength{\parindent}{0in}

\begin{document}
\begin{CJK*}{UTF8}{gbsn}
	\title{SPDE Notes based on Lototsky}
	\author{Liam Llamazares}
	\date{2022/07/09}
	\maketitle
	\section{Notation}
	Stochastic basis, i.e. filtered probability space with the usual conditions $(\Omega,\Ff,\mathbb{P})$ and $\Ff_t$ and where we take as our index set some interval $I\subset \R_+$.\\
	\\
	\section{Main points}
	\begin{itemize}
		\item White noise takes $f\in L^2(\R^d)$ as an input  and returns  the stochastic integral of the function (that is, a normal random variable). This is called a generalized function.
		\item This works in the same way both for time and space time deterministic functions.
		\item A White noise gives a Gaussian noise and vice versa.
	\end{itemize}
	\section{Introduction}
	On such a space there exists a countable family of independent Gaussian processes $G_k$ as a result we can construct explicitly a Brownian motion as
	\begin{equation}\label{Wiener}
		W(t)=\sum_{n=0}^{\infty}\qty(\int_{0}^t e_n(s) ds) G_n.
	\end{equation}
	Where $e_n$ is orthonormal basis of  $L^2(I)$.
	The fact that this is a Wiener process follows from the fact that the limit of Gaussian processes (in distribution and thus also in $L^2(\Omega)$) is Gaussian, that by Parseval's identity
	\begin{equation*}
		\E[W(t)W(s)]= \br{1_{[0,t]},1_{[0,s]}}_{L^2(\R_+)}=t \wedge s.
	\end{equation*}
	And the independence also follows by Parseval. This construction can be extended to $\R^d$ (where $d$ by contain both time and space) by considering $e_n$ to be orthonormal basis in  $L^2(\R^d)$ and setting for $u\in \R^d$
	\begin{equation*}
		W(u)=\sum_{n=0}^{\infty}\qty(\int_{[0,u]} e_n(v) dv) G_n.
	\end{equation*}
	The above is often called the \emph{Brownian sheet}.
	A term by term differentiation in \eqref{Wiener} suggests that we should have
	\begin{equation*}
		\dot{W}(t)=\sum_{n=0}^{\infty}  e_n(t)G_n.
	\end{equation*}
	However, this is problematic as, even if pointwise evaluation of $f_n$ is well defined, the series may well diverge. For this reason it will be necessary to view  $\dot{W}$ as a \emph{generalized process}.
	\begin{definition}
		Let $A\in \Bb(\R^d)$ we define \emph{white noise} $\dot{W}$ on  $L^2(A)$ as
		\begin{equation*}
			\dot{W}(f):=\sum_{n=0}^{\infty} \br{f,e_n}_{L^2(A)}G_n ;\quad f\in L^2(A)
		\end{equation*}
	\end{definition}
	We note that $\dot{W}(f)\sim \Nn\qty(0,\norm{f}_{L^2(A)})$.
	To motivate this definition we note that if we take $G_n=\int_{\R}e_n(t) dW(t)$ (they're Gaussian because they're the integral of a deterministic function which is defined as a $L^2$ limit of Gaussian variables) and define $f_n=\br{f,e_n}_{L^2(A)}$ then
	\begin{equation*}
		\int_{A}f(u) dW(u)=\sum_{n=0}^{\infty}  f_n \int_{A}e_n(u) dW(u)=\sum_{n=0}^{\infty}  f_n G_n=\dot{W}(f).
	\end{equation*}
	\begin{observation}
		The relationship between white noise and the previously defined Gaussian noise $dZ$ is that of $\dot{W}$ is a white noise on $\R^d$ then
		\begin{equation*}
			\dot{W}(B):=\dot{W}(1_B);\quad dZ(f):=\int_{\R^d} fdZ
		\end{equation*}
		This gives validity to us saying that Gaussian noise is just white noise and justifies the notation
		\begin{equation*}
			\int_{\R^d} f(u)dW(u):=\dot{W}(f).
		\end{equation*}
		In particular we have the familiar properties
		\begin{equation*}
			\br{\dot{W}(f),\dot{W}(g)}_{L^2(\Omega)}=\int_{A} f(u)g(u)du.
		\end{equation*}

	\end{observation}
	\begin{observation}
		The above construction already includes time. Given a sequence of independent Wiener processes $W_n$ one may consider time explicitly by setting
		\begin{align*}
			W(t,x)       :=\sum_{n=0}^{\infty} \qty(\int_{[0,x]} e_n(y) dy) W_n(s);\quad
			\dot{W}(t,x)  :=\sum_{n=0}^{\infty} \int_{\R_+}f_n(s) dW_n(s)
			.\end{align*}
	\end{observation}
	Where $f_n(s):=\br{f(s,\cdot ),e_n(\cdot )}_{L^2(\R^d)}$. Note that this is equivalent to the previous definition in the case $W_n(t)=\sum_{k=0}^{\infty}\qty(\int_{0}^tv_k(s)ds) G_{k,n}$  for some orthonormal basis $v_k \in L^2(I)$ and independent Gaussian variables $G_{k,n}$, as the product of orthonormal basis on $L^2(A)$ and $L^2(I)$ is an orthonormal basis on $L^2(I\times A)$.

	\section{Reproducing Kernel Hilbert Space}
	Definition from \cite{lototsky2017stochastic} page 105.
	\begin{definition}
		A generalized random field over a topological vector space $V$ is a collection of random variables  $\{X(u)\}_{u\in V}$ such that the following hold
		\begin{enumerate}
			\item \emph{Linearity}: $X(u+av)=X(u)+aX(v)$ for all  $u,v\in V$ and $a\in \R$.
			\item \emph{Continuity}: If $u_n\to u\in V$ then also $X(u_n)\xrightarrow{\mathbb{P}}X(u)$.
		\end{enumerate}
	\end{definition}

\end{CJK*}



\bibliography{biblio.bib}
Equivalent conditions for trace class operators are:

\begin{itemize}
	\item $\sqrt{|T|} $ is a Hilbert Schmidt operator.
	\item For some orthonormal basis $\sum_{n=0}^{\infty} \abs{\br{Te_n,e_n}}<\\infty$.
	\item $T$ is compact and  $\sum_{n=0}^{\infty} \lambda _n<\infty$ where $\lambda _n$ are the eigenvalues of $T$.
	\item There exists  $\lambda _n \in \R,\ell_n \in H^*$ and $y_n \in H$ bounded sequences such that
	      \begin{equation*}
		      T(x)=\sum_{n=0}^{\infty} \lambda _n\ell_n(x)y_n \quad\forall x\in H.
	      \end{equation*}

\end{itemize}

\end{document}
