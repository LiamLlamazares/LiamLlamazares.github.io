\documentclass[12pt]{article}
\special{papersize=3in,5in}
\usepackage[utf8]{inputenc}
%PACKAGES
\usepackage[colorlinks = true,
	linkcolor = blue,
	urlcolor  = black,
	citecolor = blue,
	anchorcolor = blue]{hyperref}
\usepackage[T1]{fontenc}
\makeatletter
\def\ps@pprintTitle{%
	\let\@oddhead\@empty
	\let\@evenhead\@empty
	\let\@oddfoot\@empty
	\let\@evenfoot\@oddfoot
}
\usepackage{amssymb,amsmath,physics,amsthm,xcolor,graphicx}
\usepackage[shortlabels]{enumitem}
\newtheorem{observation}{Observation}
\newtheorem{theorem}{Theorem}
\newtheorem{proposition}{Proposition}
\newtheorem{lemma}{Lemma}
\newtheorem{definition}{Definition}
\newtheorem{corollary}{Corollary}
\newcommand{\red}[1]{{\color{red}#1}}
\usepackage[colorlinks = true,
	linkcolor = blue,
	urlcolor  = black,
	citecolor = blue,
	anchorcolor = blue]{hyperref}
\usepackage{cleveref}
\bibliographystyle{elsarticle-num}
\newcommand{\A}{\mathbb{A}}\newcommand{\C}{\mathbb{C}}\newcommand{\E}{\mathbb{E}}\newcommand{\F}{\mathbb{F}}\newcommand{\K}{\mathbb{K}}\newcommand{\LL}{\mathbb{L}}\newcommand{\M}{\mathbb{M}}\newcommand{\N}{\mathbb{N}}\newcommand{\PP}{\mathbb{P}}\newcommand{\Q}{\mathbb{Q}}\newcommand{\R}{{\mathbb R}}\newcommand{\T}zzzzz\newcommand{\Z}{{\mathbb Z}}
\newcommand{\Aa}{\mathcal{A}}\newcommand{\Bb}{\mathcal{B}}\newcommand{\Cc}{\mathcal{C}}\newcommand{\Ee}{\mathcal{E}}\newcommand{\Ff}{\mathcal{F}}\newcommand{\Gg}{\mathcal{G}}\newcommand{\Hh}{\mathcal{H}}\newcommand{\Kk}{\mathcal{K}}\newcommand{\Ll}{\mathcal{L}}\newcommand{\Mm}{\mathcal{M}}\newcommand{\Nn}{\mathcal{N}}\newcommand{\Pp}{\mathcal{P}}\newcommand{\Qq}{\mathcal{Q}}\newcommand{\Rr}{{\mathcal R}}\newcommand{\Tt}{{\mathcal T}}\newcommand{\Zz}{{\mathcal Z}}\newcommand{\Uu}{{\mathcal U}}
\newcommand\restr[2]{{\left.\kern-\nulldelimiterspace #1\vphantom{\big|} \right|_{#2}}}

\newcommand{\br}[1]{\left\langle#1\right\rangle}
\pagestyle{empty}
\setlength{\parindent}{0in}

\begin{document}
\title{The Bochner Integral}
\author{Liam Llamazares}
\date{27-05-2022}
\maketitle
This is the first post in a series extending stochastic calculus to in Banach spaces. Our first topic is the Bochner integral. That is, the extension of the Lebesgue integral to functions valued in Banach spaces.
\section{Three line summary}
\begin{itemize}
	\item The Bochner integral is a way of integrating functions $f$ from a measure space to a Banach space.
	\item Like the Lebesgue integral, it is first constructed for piecewise constant functions $\Aa$ and extended continuously to the completion $\overline{\Aa}$.
	\item The completion $\Aa$ can be explicitly described as the space of functions with separable image and with finite $L^1$ norm. This naturally leads to the definition of  $L^p$ spaces.
\end{itemize}
\section{Notation}
We consider a measure space $(\Omega,\Ff,\mu )$ and a Banach space $(E,\Bb(E))$ where $\Bb$ is the Borel  sigma-algebra (that is, the smallest $\sigma $-algebra on $E$ containing all of the open sets) of $E$. We will abbreviate that $f:\Omega\to E$ is measurable as $f:\Ff\to \Bb(E)$.

\section{The Bochner integral}
Ok, lets get right to it, our goal is to define integration for functions valued in a Banach space
\begin{equation*}
	f:(\Omega,\mathcal{F},\mu)\to ((E,\norm{\cdot}),\mathcal{B}(E))
\end{equation*}
As anticipated, we first consider the class of simple functions \begin{equation*}
	\Aa=\left\{\sum_{k=1}^{n} x_k 1_{A_k}; x_k\in E\quad A_k \in \Ff\right\}.
\end{equation*}
We can define their integral quite naturally as
\begin{equation*}
	\int f d\mu=\int \sum_{k=1}^n x_k 1_{A_k} d\mu=\sum_{k=1}^n x_k\mu({A_k})
\end{equation*}
If we take equivalence classes, identifying functions that are equal  $\mu $ almost everywhere, we can define the norm
\begin{equation*}
	\norm{f}_\Aa:=\int_\Omega \norm{f} d\mu.
\end{equation*}
Then we get that integration is a linear and absolutely continuous function
\begin{equation*}
	\int_\Omega \cdot d\mu : \qty(\Aa,\norm{\cdot}_A)\to (E,\norm{\cdot}).
\end{equation*}
As, by a calculation, for all $f\in\Aa$.
\begin{equation*}
	\norm{\int_\Omega f}\leq\int_\Omega \norm{f} d\mu=\norm{f}_\Aa
\end{equation*}
Since $E$ is a Banach space, this shows that we can extend \cite{Continuousextension} integration in a unique way to the completion $\overline{\Aa}$ of $(\Aa,\norm{\cdot}_A)$.
Of course, now the key is knowing what this space $\overline{\Aa}$ is so we can figure out what kind of functions we can actually integrate. Our next definitions are motivated by this.
\begin{definition}
	We say a function $f:(\Omega,\mathcal{F})\to (E,\mathcal{B}(E))$ is strongly measurable if $f$ is measurable $f(\Omega)$ is separable.
\end{definition}
\begin{definition}
	For $1\leq p<\infty$ we define
	\begin{align*}
		\mathcal{L}^p(\Omega,\mathcal{F},\mu,E)       & =\left\{f:\Omega\to X: f \text{ is strongly measurable and } \int \norm{f}^p d\mu<\infty\right\}. \\
		\hat{\mathcal{L}}^p(\Omega,\mathcal{F},\mu,E) & =\left\{f:\Omega\to X: \int \norm{f}^p d\mu<\infty\right\}.
	\end{align*}
	We also define the semi-norms
	\begin{align*}
		\norm{f}_{L^p(\Omega\to E)}       & :=\qty(\int \norm{f}^p d\mu)^{1/p},\quad f\in\mathcal{L}^p(\Omega,\mathcal{F},\mu,E).       \\
		\norm{f}_{\hat{L}^p(\Omega\to E)} & :=\qty(\int \norm{f}^p d\mu)^{1/p},\quad f\in\hat{\mathcal{L}}^p(\Omega,\mathcal{F},\mu,E).
	\end{align*}
	Finally we take equivalence classes by the above semi-norms to obtain the metric spaces $L^p(\Omega\to E)$ and $\hat{L}^p(\Omega\to E).$
\end{definition}
An adaptation of the proof of the completion of $L^p$ spaces proves that $L^p(\Omega\to E)$ and $\hat{L}^p(\Omega\to E)$ are also complete.
\begin{proposition}
	Given a Banach space $E$, the space of $p$ integrable strongly measurable and measurable functions $L^p(\Omega\to E),\hat{L}^p(\Omega\to E)$ are Banach spaces.
\end{proposition}
\begin{proof}
	The proof is identical in both cases so we prove it only for $f\in L^p(\Omega\to E)$
	It suffices to show that if $f_n$ is such that
	\begin{equation*}
		\sum_{n=1}^\infty \norm{f_n}_{L^p(\Omega\to E)}<\infty.
	\end{equation*}
	Then there exists $f\in L^p(\Omega\to E)$ such that
	\begin{equation*}
		f=\sum_{n=1}^\infty f_n\in L^p(\Omega\to E).
	\end{equation*}
	To do so one first applies [Minkowski's] inequality for real valued functions to show that
	\begin{equation*}
		\sum_{n=1}^\infty \norm{f_n}_X\in L^p(\Omega\to\R).
	\end{equation*}
	Thus, the sum is finite almost everywhere. Since $E$ is complete we have that the above sum converges pointwise almost everywhere to some function
	\begin{equation*}
		f(\omega):=\sum_{n=1}^\infty f_n(\omega)\in E.
	\end{equation*}
	Furthermore we have that $f$ is strongly measurable as it is the limit of strongly measurable functions (this is a small exercise). Finally, by Fatou's lemma for real valued functions and the triangle inequality for norms
	\begin{multline*}
		\norm{f-\sum_{n=1}^N f_n}_{L^p(\Omega\to E)}=\norm{\sum_{n=N}^\infty f_n}_{L^p(\Omega\to E)}\leq \liminf_{M\to \infty}\norm{\sum_{n=N}^M f_n}_{L^p(\Omega\to E)}\\\leq \liminf_{M\to \infty}\sum_{n=N}^M \norm{f_n}_{L^p(\Omega\to X)}=\sum_{n=N}^\infty \norm{f_n}_{L^p(\Omega\to X)}\xrightarrow{N\to\infty} 0.
	\end{multline*}
	Which shows convergence in $L^p(\Omega\to E)$.
\end{proof}
Just as in the case of Lebesgue integrals the proof of the completeness of $L^p(\Omega\to E)$ serves to show that every convergent sequence must have a subsequence converging almost everywhere. This proposition is not necessary for the rest of the constructions, it's just a nice property to have in reserve.
\begin{proposition}
	Let $f_n\to f\in L^p(\Omega\to E)$, then there exists a subsequence $f_{n_k}$ converging to $f$ almost everywhere.
\end{proposition}
\begin{proof}
	In the proof of the above proposition we saw that for any absolutely convergent sum converges almost everywhere to its limit. Further, since  $f_n$ is Cauchy, we can extract a subsequence  $f_{n_k}$ with  $\|f_{n_k}-f_{n_{k-1}}\|\leq 2^{-k}$. By construction the sequence
	\begin{equation*}
		\sum_{k=0}^{\infty} f_{n_k}-f_{n_{k-1}},
	\end{equation*}
	is normally convergent and converges in $f$. By the above discussion we conclude the proof.
\end{proof}
Ok, so we've constructed some spaces of $p$-integrable functions, and shown that they are complete. You know where this is going, next stop is density town. In the standard construction of the Lebesgue integral it is used that every measurable function to $\mathbb{R}$ can be pointwise approximated by simple functions. One can achieve the same result for arbitrary metric spaces if the image of $f$ is separable.
\begin{lemma}
	Let $(E,d)$ be a metric space and let $f$ be strongly measurable. Then $f$ is the pointwise limit of simple functions $f_n \in \Aa$. Furthermore, \\$d_n(\omega):=d(f_n(\omega),f(\omega))$ is a non-increasing sequence for each $\omega\in\Omega$
\end{lemma}
\begin{proof}
	Consider a countable dense subset $\{e_k\}_{k=1}^\infty$ of $f(\Omega)$. Now define $\varphi:E\to E$
	\begin{equation*}
		\varphi_n(e):= e_j \text{ where }  d(e_j,e)=\min_{1\leq m\leq n} d(e,e_m).
	\end{equation*}
	And define $f_n:=\varphi_n \circ f$. A bit of thought shows that $\varphi_n$ is continuous and thus so is $f_n$ (remember we are considering the Borel $\sigma $-algebra) on $E$.
	Furthermore, since for each fixed $\omega$ the above distance goes to $0$ we have that
	\begin{equation*}
		\lim_{n\to\infty} f_n(\omega)=f(\omega),\quad \forall \omega\in\Omega.
	\end{equation*}
	Finally, $f_n$ is simple as $f_n(\Omega)\in \{e_1,...,e_n\}$ and the non-increasing property of $d_n(\omega)$ is clear by construction.
\end{proof}
In the above proof we see that the reason for requiring that the image of our class of integrable functions be separable is so that we can approximate them by simple functions.
\begin{proposition}
	Every function $f\in L^1(\Omega\to E)$ is the limit of simple functions.
\end{proposition}
\begin{proof}
	Since $f$ is strongly measurable, we can apply the above lemma to obtain a sequence of simple functions $f_n$ converging pointwise and monotonically to $f$. Furthermore we have that, by the monotone convergence theorem for integrals in $\mathbb{R}$,
	\begin{equation*}
		\lim_{n\to\infty}\norm{f_n-f}_{L^1(\Omega\to E)}=\lim_{n\to\infty}\int_\Omega\norm{f_n-f}d\mu=\int_{\Omega}\lim_{n \to \infty}\|f_n-f\| d\mu =0.
	\end{equation*}
\end{proof}
As a corollary we obtain the following
\begin{corollary}
	$\mathcal{A}$ is a dense subset of $\qty(L^1(\Omega\to E),\norm{\cdot}_{L^1(\Omega\to E)})$.
\end{corollary}
In consequence, since we already saw that $L^1(\Omega\to E)$ is complete, we have that $\overline{\Aa}= L^1(\Omega\to E)$. This is exactly the space of integrable functions.
\begin{definition}
	We define the integral on $L^1(\Omega\to E)$ as the unique continuous extension with the norm $\norm{\cdot}_{L^1(\Omega\to E)}$ of the integral on $\mathcal{A}$. That is, given $f\in L^1(\Omega\to E)$ we define
	\begin{equation*}
		\int_\Omega f d\mu:=\lim_{n\to\infty} \int_\Omega f_n d\mu.
	\end{equation*}
	Where $f_n\in \mathcal{A}$ is any sequence such that $\norm{f-f_n}_{L^1(\Omega\to E)}\to 0$.
\end{definition}
The typical properties are now just a result of the definition and a passage to the limit, as they hold for simple functions.
\begin{corollary}
	Let $f\in L^1(\Omega,E)$ with $E$ a Banach space, then
	\begin{enumerate}
		\item $\norm{\int_\Omega f d\mu}\leq\int_\Omega{\norm{f}d\mu} $
		\item Let $Y$ be another Banach space and $L\in L(E,Y)$. Then
		      \begin{equation*}
			      \int_\Omega (L\circ f) d\mu=L\qty(\int_\Omega f d\mu).
		      \end{equation*}
	\end{enumerate}
\end{corollary}
Ok, that's it, this post was a bit more technical than some of the others but you get the picture. Define an integral for simple functions, and figure out what can be approximated by simple functions. As we saw, the extra requirement that appears over the Lebesgue case is that the function $f$ is separately valued and justifies why, as we will see in future posts on SPDEs, the image of $f$ is often taken to be some separable Hilbert space. Until the next time!
\bibliography{biblio.bib}
\end{document}
